\documentclass[a4paper, 14pt]{article}

%%% Матпакет
\usepackage{amsmath, amssymb, amscd, amsthm, amsfonts}
\usepackage{hyperref}
\usepackage{icomma}                  % "Умная" запятая: $0,2$ --- число, $0, 2$ --- перечисление

%%% Страница
\usepackage{extsizes} % Возможность сделать 14-й шрифт
\usepackage{geometry} % Простой способ задавать поля
	\geometry{top=25mm}
	\geometry{bottom=25mm}
	\geometry{left=18mm}
	\geometry{right=14mm}
\usepackage{indentfirst}

%%% Стили
\usepackage{xcolor}
%%\usepackage{sectsty}
%%\allsectionsfont{\sffamily}
\usepackage{titlesec, blindtext, color} % подключаем нужные пакеты
\definecolor{gray75}{gray}{0.44} % определяем цвет
\definecolor{bolotozeleny}{RGB}{41, 176, 107}
\newcommand{\hsp}{\hspace{14pt}} % длина линии в 20pt
\titleformat{\section}[hang]{\Large\bfseries}{\thesection\hsp\textcolor{gray75}{|}\hsp}{0pt}{\Large\bfseries\textcolor{bolotozeleny}}

%%% Работа с русским языком
\usepackage{cmap}					% поиск в PDF
\usepackage{mathtext} 				% русские буквы в фомулах
\usepackage[T2A]{fontenc}			% кодировка
\usepackage[utf8]{inputenc}			% кодировка исходного текста
\usepackage[english, russian]{babel}	% локализация и переносы

\usepackage{soul} % Модификаторы начертания
\usepackage{csquotes} % Цитаты

%%% Теоремы
\theoremstyle{plain} % Это стиль по умолчанию, его можно не переопределять.
\newtheorem{theorem}{Теорема}[section]
\newtheorem{proposition}[theorem]{Утверждение}
\newtheorem{definition}{Определение}

\theoremstyle{definition} % "Утверждение"
\newtheorem*{solution}{Решение}
\newtheorem{problem}{Задача}[subsection]

\theoremstyle{remark} % "Примечание"
\newtheorem{example}{Пример}
\newtheorem{nota}{Примечание}

%%% Работа с картинками
\usepackage{graphicx}                % Для вставки рисунков
\graphicspath{{img/}}  				% папки с картинками
\setlength\fboxsep{3pt}              % Отступ рамки \fbox{} от рисунка
\setlength\fboxrule{1pt}             % Толщина линий рамки \fbox{}
\usepackage{wrapfig}                 % Обтекание рисунков текстом
\title{Корпоративные финансы \\ Решение задач}
\author{ЭФ МГУ}
\date{осень - зима 2020 г.}

%%% Работа с таблицами
\usepackage{array,tabularx,tabulary,booktabs} % Дополнительная работа с таблицами
\usepackage{longtable}                        % Длинные таблицы
\usepackage{multirow}                         % Слияние строк в таблице

\usepackage{hyperref}
\usepackage[usenames, dvipsnames, svgnames, table,rgb]{xcolor}
\hypersetup{				                % Гиперссылки
    unicode=true,                           % русские буквы в раздела PDF
    pdftitle={Задачи},                      % Заголовок
    pdfauthor={Neqqi},                      % Автор
    pdfsubject={КорпФин},                   % Тема
    pdfcreator={Neqqi},                     % Создатель
    pdfproducer={Neqqi},                    % Производитель
    pdfkeywords={keyword1} {key2} {key3},   % Ключевые слова
    colorlinks=true,        	            % false: ссылки в рамках; true: цветные ссылки
    linkcolor=magenta,                          % внутренние ссылки
    citecolor=green,                        % на библиографию
    filecolor=magenta,                      % на файлы
    urlcolor=cyan                           % на URL
}

\begin{document}
\maketitle

\section{Современная финансовая модель компании}

\section{Инвестиционные решения компании относительно реальных активов}
\subsection{Задачи на расчет NWC}

\begin{problem}
    Компания «Модерн», выпускающая мебель, планирует строительство нового цеха по производству офисной мебели. Для этого сейчас, в году $n = 0$, компания осуществит инвестиции в покупку оборудования на сумму 315 млн. руб. Кроме того, в результате реализации проекта в нулевом году у компании увеличатся некоторые статьи баланса. В частности,  запасы  материалов  возрастут  на  40  млн.  руб.,  запасы  комплектующих увеличатся  на  35  млн.  руб.,  а  дебиторская  задолженность  повысится  на 32  млн.  руб.Одновременно  у  компании  увеличатся  расходы  будущих  периодов  на  8  млн.  руб., кредиторская задолженность станет больше на 60 млн. руб. Рассчитайте \textbf{инвестиции в чистый оборотный капитал} компании в нулевом году.
    \begin{solution}
        Инвестиции в чистый оборотный капитал рассчитываются по формуле:
        \[NWC_{INV} = \Delta CurrAssets - \Delta CurrLiab \]
        Покупка оборудования $\subseteq$ инвестиции в долгосрочные активы $\Longrightarrow$ исключаем статью из анализа (расчет проводится по краткосрочным активам и пассивам)

        Построим таблицу текущих активов и пассивов:

        \vspace{0.4em}
        \begin{tabularx}{0.98\textwidth}{|c|X|c|X|}
            \hline
            \textbf{Текущие активы} & \textbf{млн.руб.} & \textbf{Текущие обязательства} & \textbf{млн.руб.} \\
            \hline
            Запасы материалов & 40 & Кредиторская задолженность & 60 \\
            \hline
            Запасы комплектующих & 35 & - & - \\
            \hline
            Дебиторская задолженность & 32 & - & - \\
            \hline
            Расходы будущих периодов & 8 & - & - \\
            \hline
            \textbf{Итого текущие активы} & \textbf{115} & \textbf{Итого текущие обязательства} & \textbf{60} \\
            \hline
        \end{tabularx}

        \vspace{0.4em}
        Подставим в формулу:
        \[NWC_{INV} = \text{115 млн.руб.} - \text{60 млн.руб.} = \textbf{55 млн.руб} \]
    \end{solution}
\end{problem}

\begin{problem}
    Компания «Объектив» реализует проект по выпуску телевизоров с технологией 8К. В год t = 0 компания потратит 75000 млн руб. на закупку оборудования и инвестирует в чистый  оборотный  капитал  2000  млн  руб.  Проект  будет  реализован  в  течение последующих  5-ти  лет,  при  этом  отдел  маркетинга  подготовил  данные  о  выручке,  а бухгалтерия – о себестоимости продукции:

    \vspace{0.4em}
    \begin{tabularx}{0.88\textwidth}{|X|c|c|c|c|c|}
        \hline
        \textbf{Год (порядковый номер)} & \textbf{1} & \textbf{2} & \textbf{3} & \textbf{4} & \textbf{5} \\
        \hline
        Продажи, млн.руб. & 34000 & 40000 & 35000 & 28000 & 25000 \\
        \hline
        Себестоимость, млн.руб. & 17000 & 20000 & 17500 & 14000 & 12500 \\
        \hline
    \end{tabularx}

    \vspace{0.4em}
    Известно  также,  что  в  период  реализации  проекта в  конце  года t = 1 баланс дебиторской задолженности составит 30\% от выручки проекта, величина запасов составит 15\%  от  себестоимости,  а  кредиторская  задолженность  40\%  от  себестоимости.  Все инвестиции в чистый оборотный капитал будут восстановлены в году t = 5. Определите \textbf{величину  инвестиций  в  чистый  оборотный  капитал}  на  период  реализации  проекта, необходимые для расчета денежных потоков данного проекта?

    \begin{solution}
        Основная формула та же:
        \[NWC_{INV} = \Delta CurrAssets - \Delta CurrLiab \]
        Закупка оборудования $\subseteq$ инвестиции в Long-term активы $\Longrightarrow$ исключаем статью из анализа. А вот инвестиции в основной капитал в объеме 2000 млн.руб. --- включаем в анализ.

        Используем условия для расчета:
        \[\textit{Дебиторская задолженность} = 0.3\times\textit{Выручка}\]
        \[\textit{Кредиторская задолженность} = 0.4\times\textit{Себестоимость}\]
        \[\textit{Запасы} = 0.15\times\textit{Себестоимость}\]
        Построим полную таблицу и запишем данные:

        \vspace{0.4em}
        \begin{tabularx}{0.88\textwidth}{|X|c|c|c|c|c|c|}
            \hline
            \textbf{Год (порядковый номер)} & \textbf{0} & \textbf{1} & \textbf{2} & \textbf{3} & \textbf{4} & \textbf{5}\\
            \hline
            Продажи, млн.руб. & - & 34000 & 40000 & 35000 & 28000 & 25000 \\
            \hline
            Себестоимость, млн.руб. & - & 17000 & 20000 & 17500 & 14000 & 12500 \\
            \hline
            Дебиторская задолженность & - & 10200 & 12000 & 10500 & 8400 & 7500 \\
            \hline
            Кредиторская задолженность & - & 6800 & 8000 & 7000 & 5600 & 5000 \\
            \hline
            Запасы & - & 2550 & 3000 & 2625 & 2100 & 1875 \\
            \hline
            NWC & -2000 & -5950 & -7000 & -6125 & -4900 & -4375 \\
            \hline
            \Delta \textbf{NWC} & \textbf{-2000} & \textbf{-3950} & \textbf{-1050} & \textbf{875} & \textbf{1225} & \textbf{4900} \\
            \hline
        \end{tabularx}
            \vspace{0.4em}

            Наконец, рассчитаем изменение NWC $(\forall t={1, 2, 3, 4})$ как:
            \[\Delta NWC_{t} = NWC_{t} - NWC_{t-1}\]

            Для $t = 5$ суммируем:
            \[\sum_{t=1}^4 \Delta NWC_{t} = \textbf{4900 млн.руб.} \]

			\begin{nota}
				В последнем году инвестиции в чистый оборотный капитал, связанные с проектом, полностью возмещаются. Нужно просуммировать все инвестиции, включая нулевой период и последующие четыре года, и отразить возврат NWC со знаком плюс в последнем году.
			\end{nota}
        \end{solution}
\end{problem}

\subsection{Задачи на расчет денежных потоков по новому проекту}

\begin{problem}
    Компания  «Северное  море»  в  рамках  стратегии  увеличения  доли  рынка производства  и  добычи  рыбных  биоресурсов,  реализует  проект  «Аквакультура»  по выращиванию  форели  и  семги.  В  текущем  году  (n = 0)  предполагается  закупка  нового оборудования. Стоимость нового оборудования составляет 675 млн. руб., при этом его доставка из Европы составит 22 млн., наладка и монтаж обойдутся в 23 млн. руб. Срок службы нового оборудования составляет 8 лет. Учетная политика компании предполагает линейный метод начисления амортизации по оборудованию. Проект рассчитан на 5 лет и со  следующего  года  планируется  его  начало. Первоначальные  инвестиции  в  чистый оборотный капитал в нулевой период составят  30\% от объема выручки 1-го года, а в дальнейшем в конце каждого следующего года инвестиции в NWC будет увеличиваться на  сумму,  равную 10\%  прироста  выручки,  ожидаемого  в  следующем  году. Преимуществом проекта является более высокая свежесть рыбы (доставляется за 1-2 дня из  Карелии,  в  отличие  от  5-6  дней  из  Норвегии),  в  результате  спрос  на  продукцию компании  возрастет,  и  выручка  составит  1000  млн.  руб.  в  первый  год,  а  в  каждый последующий год будет возрастать на 20\%. По новому проекту переменные затраты на корм  для  рыбы  составят  30\%  выручки,  затраты  на  энергию  и  труд –25\%  выручки. Постоянные расходы не будут изменяться в течение срока реализации проекта и составят 50  млн.  руб.  в  год.  В  результате  реализации  проекта  компания  обеспечит  себя собственным  сырьем  для  производства  рыбных  деликатесов  (ранее  закупалось  в Норвегии), что обеспечитей дополнительную экономию расходов в объеме 320 млн. руб. ежегодно. Новый проект рассчитан на 5 лет, и в конце этого срока новое оборудование будет продано за 290 млн. руб. Компания платит налог на прибыль по ставке 25\%. Средневзвешенные затраты на капитал компании составляют 15\%. Рассчитайте \textbf{NPV} для предложенного проекта.
    \begin{solution}
		Сначала рассчитаем первоначальные инвестиции, которые связаны с покупкой оборудования, а именно – первоначальную стоимость закупаемых основных средств:
		\[FC_{INV} = \textit{Стоимость Оборудования} + \textit{Наладка} + \textit{Монтаж} = \textbf{720 млн.руб.} \]

		Используя условие задачи, составим формулы:
		\[NWC_{INV}^{t=0} = 0.3 \times \textit{Выручка}^{t=1} = \textbf{300 млн.руб.} \]
		\[NWC_{INV}^{t} = NWC_{INV}^{t-1} + 0.1\times(\textit{Выручка}^{t+1}-\textit{Выручка}^{t})\]
		\[\textit{Выручка}^{t} = \text{1000 млн.руб} \times 0.2^{t-1}\]
		\[VC_{fish} = 0.3 \times \textit{Выручка}\]
		\[VC_{energy} = 0.25 \times \textit{Выручка}\]
		\[CF_{0} = FC_{INV} + NWC_{INV}^{0} = \textbf{1020 млн.руб.}\]

		Рассчитаем амортизацию и балансовую стоимость:
		\[D_{t} = \frac{FC_{INV}}{8} = \textbf{90 млн.руб.}\]
		\[B_{t} = FC_{INV} - t \times D_{t}\]
		\[B_{5} = FC_{INV} - 5 \times D_{t} = \textbf{270 млн.руб.}\]

		Рассчитаем операционную прибыль (EBIT) и прибыль после уплаты налогов:
		\[EBIT = \textit{Выручка} + VC + FC + \textit{Экономия} - D_{t}\]
		\[NOPLAT = EBIT \times (1-Taxrate)\]
\newpage
\begin{table}
	\centering
	\begin{tabular}[0.9\textwidth]{|c|c|c|c|c|c|c|}
		\hline
		\textbf{t}                            & \textbf{0}           & \textbf{1}           & \textbf{2}           & \textbf{3}           & \textbf{4}           & \textbf{5}            \\
		\hline
		Выручка                      &             & 1000        & 1200        & 1440        & 1728        & 2073.6       \\
		\hline
		Переменные затраты (корм)   &             & -300        & -360        & -432        & -518.4      & -622.08      \\
		\hline
		Переменные затраты (энергия) &             & -250        & -300        & -360        & -432        & -518.4       \\
		\hline
		Постоянные затраты                   &             & -50         & -50         & -50         & -50         & -50          \\
		\hline
		Экономия расходов            &             & 320         & 320         & 320         & 320         & 320          \\
		\hline
		Амортизация                 &             & -90         & -90         & -90         & -90         & -90          \\
		\hline
		Операционная прибыль         &             & 630         & 720         & 828         & 957.6       & 1113.12      \\
		\hline
		Прибыль после уплаты налога  &             & 472.5       & 540         & 621         & 718.2       & 834.84       \\
		\hline
		Корректировка на амортизацию &             & 90          & 90          & 90          & 90          & 90           \\
		\hline
		\Delta \text{NWC}               &             & -20         & -24         & -28.8       & -34.56      & 407.36       \\
		\hline
		NRV                          &             &             &             &             &             & 285          \\
		\hline
		CF                           & -1020       & 542.5       & 606         & 682.2       & 773.64      & 1617.2       \\
		\hline
		DCF                          & -1020       & 471.73 & 458.22 & 448.55 & 442.33 & 804.03  \\
		\hline
		\textbf{NPV}                          & \textbf{1604.88} &             &             &             &             &              \\
		\hline
	\end{tabular}
\end{table}

Рассчитаем CF, скорректировав прибыль после уплаты налогов на амортизацию и изменение NWC. Рассчитаем DCF, используя ставку средневзвешенных затрат на капитал (WACC):
\[CF = NOPLAT + D_{t} + \Delta NWC \]
\[DCF_{t} = \frac{CF_{t}}{1 + WACC^{t}}\]
\begin{nota}
	Для последнего года учитываем также NRV!
\end{nota}
\[NRV = Price - Taxrate \times (Price - B_{5})\]
Рассчитаем NPV, просуммировав DCF:
	\[NPV = \sum_{t=0}^{5} DCF_{t} \]
	\[NPV = \textbf{1604.88 млн.руб.}\]

Так как NPV положительный, принимается решение \textbf{принять проект}!
    \end{solution}
\end{problem}

\begin{problem}
    Крупный  химический  завод  «Передовая  химия»  решает  вопрос  о  реализации нового проекта по производству перекиси водорода новым перспективным способом. Для этого сейчас, в году t = 0, компания закупит оборудование на 2385 млн руб., его доставка обойдется в 10 млн руб., пуско-наладочные работы –в 35 млн руб. Срок службы данного оборудования составляет 9 лет. Предполагается, что все потоки по проекту поступают в конце  года. Учетная  политика  компании  предполагает  линейный  метод  начисления амортизации. Первоначальные инвестиции в чистый оборотный капитал в нулевой период составят 15\% от объема выручки 1-го года, а в дальнейшем в конце каждого следующего года инвестиции в NWC будет увеличиваться на сумму, равную 5\% прироста выручки, ожидаемого в следующем году. (...)
	Оцените \textbf{целесообразность реализации} проекта и рассчитайте \textbf{NPV}.
    \begin{solution}
		Аналогично задаче выше.
    \end{solution}
\end{problem}

\subsubsection{Задача на расчет денежных потоков по новому проекту в случае возникновения эффекта каннибализации.}

\begin{problem}
    \begin{solution}

    \end{solution}
\end{problem}

\subsection{Задачи на расчет денежных потоковв случае замены оборудования}

\begin{problem}
	Крупная  авторемонтная  компания  «Самоделкин»  4  года  назад  приобрела оборудование за 135 млн. руб. для ремонта двигателей автомобилей. Срок службы данного оборудования составляет 9 лет. Сейчас оно может быть продано на рынке за 50 млн. руб. Вместе с тем, руководство компании понимает, что конструкции двигателей постоянно усложняются, поэтому для повышения качества обслуживания потребителей необходимо новое оборудование, которое позволит повысить точность обработки деталей двигателей. Компания рассматривает вариант продажи старого оборудования и покупки нового за 180 млн.  руб.,  доставка  и  монтаж которого обойдутся  в  4  млн.  руб.  Срок  службы  нового оборудования  составляет  8  лет.  Учетная  политика  компании  предполагает  линейное начисление амортизации как по старому, так и по новому оборудованию. Новый проект рассчитан на 5 лет, после чего станки можно будет продать за 40 млн. руб. Предполагается, что, начиная с 3-го года, ежегодные затраты на техническое обслуживание станков будут составлять  15  млн.  руб.  С  новым  оборудованием  появится  возможность  быстрее обрабатывать  детали,  станки  будут  более  универсальными.  В  результате  выручка компании  возрастет  с  100  млн.  руб.  до  195  млн.  руб.  ежегодно.  Кроме  того,  замена оборудования  позволит  снизить  долю  переменных  расходов  в  выручке  и  величину постоянных расходов:
	\begin{center}
	\begin{tabular}{|c|c|c|}
	\hline
	                    & VC               & FC            \\
	\hline
	Старое оборудование & 60\% в выручке:  & 48 млн. руб.  \\
	\hline
	                    & Зарплата – 30\%  &               \\
	\hline
	                    & Энергия – 15\%   &               \\
	\hline
	                    & Материалы – 15\% &               \\
	\hline
	Новое оборудование  & 50\% в выручке:  & 17 млн. руб.  \\
	\hline
	                    & Зарплата – 25\%  &               \\
	\hline
	                    & Энергия – 10\%   &               \\
	\hline
	                    & Материалы – 15\% &               \\
	\hline
	\end{tabular}
	\end{center}
	Замена  оборудования  на  момент  его закупки  потребует  дополнительных инвестиций  в  запасы  в  размере  10  млн.  руб.  Предполагается,  что  любое  движение денежных средств происходит в конце периода. Компания платит налог на прибыль по ставке 24\%.  Средневзвешенные  затраты  на  капитал  компании  оцениваются  в  10\%. Рассчитайте NPV проекта.

    \begin{solution}
		Сначала определим величину амортизации старого и нового оборудования, которая начисляется линейным методом:
		\[D_{t}^{0} = \frac{\text{135 млн.руб.}}{9} = \textbf{15 млн.руб.}\]
		\[D_{t}^{1} = \frac{\text{184 млн.руб.}}{8} = \textbf{23 млн.руб.}\]

		Балансовая стоимость старого оборудования на момент закупки нового:
		\[B_{0} = FC_{INV} - 4 \times D_{t} = \textbf{75 млн.руб.}\]

		Стоимость продажи, налоговый щит (TS) и чистая реализуемая стоимость старого оборудования:
		\[Price_{0} = \textbf{50 млн.руб.}\]
		\[TS_{0} = 0.24\times(B_{0} - Price_{0}) = \textbf{6 млн.руб.}\]
		\[NRV_{0} = Price_{0} + TS_{0} = \textbf{56 млн.руб.}\]

		При этом $NWC_{INV} = \textbf{10 млн.руб.}$.
		А значит
		\[CF_{0} = NRV_{0} + FC_{INV} + NWC_{INV} = \textbf{138 млн.руб.}\]

		Стоимость продажи, налоговый щит (TS) и чистая реализуемая стоимость нового оборудования:
		\[Price_{1} = \textbf{40 млн.руб.}\]
		\[TS_{1} = 0.24\times(B_{1} - Price_{1}) = \textbf{22.56 млн.руб.}\]
		\[NRV_{1} = Price_{1} + TS_{1} = \textbf{62.56 млн.руб.}\]

		Выгода от сокращения постоянных расходов составит \textbf{31 млн.руб.}, а издержки от увеличения переменных --- \textbf{37.5 млн.руб.}. Увеличение выручки составит \textbf{95 млн.руб.}, увеличение амортизации оборудования --- \textbf{8 млн.руб.}.
		На основании этих данных строим таблицу:

\vspace{0.4em}
\begin{tabular}{|l|l|l|l|l|l|l|}
\hline
                              & \textbf{0}      & \textbf{1}      & \textbf{2}      & \textbf{3}      & \textbf{4}      & \textbf{5}       \\
\hline
Изменение выручки             &        & 95     & 95     & 95     & 95     & 95      \\
\hline
Выгода от сокращения FC       &        & 31     & 31     & 31     & 31     & 31      \\
\hline
Выгода от сокращения VC       &        & -37.5  & -37.5  & -37.5  & -37.5  & -37.5   \\
\hline
Техническое обслуживание      &        &        &        & -15    & -15    & -15     \\
\hline
Амортизация                   &        & -8     & -8     & -8     & -8     & -8      \\
\hline
Операционная прибыль          &        & 80.5   & 80.5   & 65.5   & 65.5   & 65.5    \\
\hline
Прибыль после налогообложения &        & 61.18  & 61.18  & 49.78  & 49.78  & 49.78   \\
\hline
Корректировка на амортизацию  &        & 8      & 8      & 8      & 8      & 8       \\
\hline
возмещение NWC                &        &        &        &        &        & 10      \\
\hline
NRV для нового оборудования   &        &        &        &        &        & 62.56   \\
\hline
СF                            & -138       & 149.68 & 149.68 & 123.28 & 123.28 & 195.84  \\
\hline
DCF                           & -138       & 136.07 & 123.70 & 92.62 & 82.20 & 121.60  \\
\hline
\textbf{NPV}                           & \textbf{420.2} &        &        &        &        &         \\
\hline
\end{tabular}

\vspace{0.4em}
Рассчитываем все показатели аналогично предыдущим задачам, получаем CF, DCF и $\textbf{NPV} = \textbf{420.2 млн.руб.}$
    \end{solution}
\end{problem}

\subsubsection{Задача  на  расчет денежных  потоковв  случае  замены  оборудования  и инвестициями в 3-м году}

\begin{problem}
    \begin{solution}

    \end{solution}
\end{problem}
\newpage

\subsubsection{Анализ проектов с неравными сроками.}
\begin{problem}
	В компании рассматриваются два альтернативных взаимоисключающих проекта с разными сроками реализации. При этом проект А рассчитан на 5 лет, а проект В – на 10 лет. Известны данные по ежегодным прогнозным денежным потокам данных проектов:
\begin{center}
\begin{tabular}[0.88\textwidth]{|p{4cm}|l|l|l|l|l|l|l|l|l|l|l|}
\hline
Год:                               & 0    & 1   & 2   & 3   & 4   & 5   & 6   & 7   & 8   & 9   & 10   \\
\hline
Денежный поток проекта А, млн.руб. & -400 & 120 & 390 & 500 & 470 & 420 &     &     &     &     &      \\
\hline
Денежный поток проекта B, млн.руб. & -750 & 230 & 280 & 320 & 460 & 500 & 580 & 700 & 600 & 550 & 500  \\
\hline
\end{tabular}
\end{center}

Средневзвешенные затраты на капитал компании равны 15\%.
\begin{itemize}
	\item[a)] Рассчитайте NPV проектов на основе метода цепного повтора и сделайте вывод о предпочтительности проекта.
	\item[b)] Сравните NPV проектов на основе метода эквивалентного аннуитета.
\end{itemize}
	\begin{solution}
		Для осуществления метода цепного повтора определяем НОК число лет. В данном случае это 10. В таком случае осуществляем проект А два раза и подсчитываем общее NPV.
\begin{center}
\begin{tabular}[0.6\textwidth]{|p{0.74cm}|l|l|l|l|l|l|l|l|l|l|l|}
\hline
Год:                               & 0       & 1      & 2      & 3      & 4      & 5      & 6      & 7      & 8      & 9      & 10      \\
\hline
CF А & -400    & 120    & 390    & 500    & 470    & 20     & 120    & 390    & 500    & 470    & 420     \\
\hline
CF B & -750    & 230    & 280    & 320    & 460    & 500    & 580    & 700    & 600    & 550    & 500     \\
\hline
DCF A                      & -400    & 104.35 & 294.89 & 328.75 & 268.72 & 9.94   & 51.87  & 146.61 & 163.45 & 133.60 & 103.82  \\
\hline
DCF B                      & -750    & 200    & 211.72 & 210.40 & 263.01 & 248.58 & 250.75 & 263.16 & 196.14 & 156.34 & 123.59  \\
\hline
\end{tabular}
\end{center}
Суммируем дисконтированные потоки для 2 проектов А и 1 проекта B и получаем: $NPV_{A2} = 1206.04$ (первый проект), $NPV_{B} = 1373.70$ (второй проект). Выбираем второй.

\vspace{1em}
\textbf{Метод эквивалентного аннуитета:}
	\[NPV_{A} = \textbf{805.54 млн.руб.} \]
	\[NPV_{B} = \textbf{1373.70 млн.руб.} \]

Рассчитаем платеж эквивалентного аннуитета для каждого из сравниваемых проектов:
	\[EAA_{i} = \frac{NPV_{i} \times r}{1-(1+r)^{-T_{i}}}\]
	\[EAA_{A} = \textbf{240.31 млн.руб.}\]
	\[EAA_{A} = \textbf{273.71 млн.руб.}\]

Рассчитываем стоимость бессрочного аннуитета для каждого проекта, платеж по которому равен платежу по соответствующему эквивалентному аннуитету, по следующей формуле:
	\[NPV_{i}^{\infty} = \frac{EAA_{i}}{r}\]
	\[NPV_{A}^{\infty} = \textbf{1602.03 млн.руб.}\]
	\[NPV_{A}^{\infty} = \textbf{1824.75 млн.руб.}\]

Смотрим и выбираем, опять же, проект B.
	\end{solution}
\end{problem}

\section{Выбор источников финансирования компании.}

\section{Расчет и анализ средневзвешенных затрат на капитал (WACC)}

\subsection{Задача на оценку затрат на собственный капитал.}
\begin{problem}
		Корпорация «Трамвай удачи» выпускает несколько новых усовершенствованных моделей  бесшумных  и  удобных  в эксплуатации  трамваев  как  экологически  чистого транспорта для крупных городов, конкурирует со многими отечественными и зарубежными компаниями. О корпорации известна следующая информация:
		\begin{itemize}
			\item[---] доходность облигаций компании с рейтингом АА равна 8.9\%, превышение доходности акций компании над облигациями с аналогичным рейтингом составляет 6.1\%;
			\item[---] рыночная премия за риск 6,4%;
			\item[---] β-коэффициент компании 1,4;
			\item[---] компания является стабильно растущей, а ожидаемый темп прироста дивидендов предполагается 6,8\%;
			\item[---] безрисковая ставка процента, рассчитанная на основе доходности государственных облигаций со сроком обращения 1 год равна 5.2\%, со сроком обращения 5 лет --- 5.9\%, со сроком обращения 30 лет --- 6.3%;
			\item[---] дивиденд на акцию в конце года составит 12 руб.
			\item[---] текущая рыночная стоимость акций составляет 150 руб.
			\item[---] корпорация облагается налогом по ставке 25%;
		\end{itemize}

		Согласно расчетам  аналитиков,  за  риск  инвестиций  в  данную  компанию  нужно начислить следующие премии:
		\begin{itemize}
			\item премия за размер компании 0,5%;
			\item премия за товарную диверсификацию 0,5%;
			\item премия за территориальную диверсификацию 1%;
			\item премия за диверсификацию клиентуры 0,4%;
			\item премия за финансовый риск компании 0,6%;
		\end{itemize}

		Для корпорации «Трамвай удачи» рассчитайте затраты на собственный капитал на основе трех моделей:
		\begin{itemize}
			\item[a:] модель оценки долгосрочных активов (САРМ);
			\item[b:] доходность облигации плюс премия за риск;
			\item[c:] дисконтированного денежного потока;
			\item[d:] метод кумулятивного построения.
		\end{itemize}
		Какой  из  трех  предложенных  подходов  предпочтительнее  использовать? Аргументируйте свой ответ. Можно ли в этом случае для оценки затрат на собственный капитал использовать среднюю из полученных значений? Поясните свою точку зрения.
	\begin{solution}
	\end{solution}
		\begin{enumerate}
			\item \textbf{CAPM:} $r_{e} = r_{f} + \beta \times (r_{m} - r_{f})$

			$r_{e} = 6.3\% + 1.4 \times 6.4\% = \textbf{15.26\%}$
			\item \textbf{«Доходность облигаций компании + премия за риск»:} $r_{e} = r_{d} + \textit{Премия за риск}$

			$r_{e} = 8.9\% + 6.1\% = \textbf{15\%}$
			\item \textbf{DCF:} $r_{e} = \frac{div}{P_{s}+g} $

			$r_{e} = \frac{12}{150+0.068} = \textbf{14.8\%} $
			\item \textbf{Метод кумулятивного построения:} $r_{e} = r_{f} + (r_{m} - r_{f}) + \sum_{i=1}^J G_{i}$

			$r_{e} = 6.3\% + 6.4\% + 0.5\% + 0.5\% + 1\% + 0.4\% + 0.6\% = \textbf{15.7\%}$
		\end{enumerate}

		Предпочтительнее использовать САРМ, т.к. этот показатель дает рыночную оценку. Здесь все данные примерно равны, и поэтому можно использовать среднюю из них:
		\[r_{e} = \frac{15.26\% + 15\% + 14.8\% + 15.7\%}{4} = \textbf{15,19\%}\]

		Если методы дают сильно различающиеся данные, то финансовому менеджеру следует, исходя из здравого смысла, оценить достоинства и недостатки каждой оценки и выбрать ту из них, которая кажется наиболее достоверной при данных условиях. В значительной степени этот выбор основан на понимании финансовым менеджером сущности и роли исходных параметров в каждом методе.
\end{problem}

\begin{problem}
	Компания «Московская газель» выпускает микроавтобусы, и имеет следующий упрощенный баланс,млн. руб.:

\begin{center}
\begin{tabular}{|l|l|l|l|}
\hline
Основные средства         & 500 & Обыкновенные акции              & 200   \\
\hline
Запасы                    & 120  & Добавочный капитал              & 22    \\
\hline
Дебиторская задолженность & 330  & Нераспределенная прибыль        & 188   \\
\hline
Денежные средства         & 232  & Привилегированные акции                       & 100   \\
\hline
                          &      & Долгосрочный банковский кредит  & 260   \\
\hline
                          &      & Краткосрочный банковский кредит & 150   \\
\hline
                          &      & Кредиторская задолженность      & 252   \\
\hline
\textbf{Итого}                     & \textbf{1182} &                                 & \textbf{1182}  \\
\hline
\end{tabular}
\end{center}
\begin{itemize}
	\item Краткосрочные банковские кредиты компания использует не на постоянной основе.
	\item Долгосрочный банковский кредит будет погашаться равными долями в размере 53,406 млн. руб. в течение 7 лет.
	\item Привилегированные акции номинальной стоимостью 100 руб. сейчас продаются на рынке по 110 руб. В обращении находится 1 млн. привилегированных акций. Фиксированный дивиденд установлен в размере 13,31\% к номиналу.
	\item Сейчас в обращении на рынке находится 4 млн. обыкновенных акций, и одна акция продается по 157,5 руб. (номинал акции 50 руб.).В прошлом году акции компании котировались по 180 руб.
	\item Последний выплаченный дивиденд на акцию компании составил 15 руб.
	\item Рентабельность собственного капитала компании составляет 12,5\%, и ежегодно корпорация реинвестирует в производство 40\% своей прибыли;
	\item $\beta$-коэффициент компании оценивается в 1,1. Ставка налога на прибыль компании –20\%. Рассчитайте средневзвешенные затраты на капитал для данной компании.
\end{itemize}
	\begin{solution}
		Сначала рассчитаем рыночную стоимость отдельных источников финансирования и общий объем финансирования.
	\begin{itemize}
		\item[(а)] Собственный капитал (по рыночной оценке):
		\[Equity = Q_{s} \times P_{s} + \textit{Добавочный капитал} + \textit{Нераспределенная прибыль}\]
		\[Equity = 4 \times 10^{6} \times \text{157.5 руб.} +  \text{22 млн.руб.} +  \text{188 млн.руб.} = \textbf{840 млн.руб.}\]
		\item[(b)] Привилегированные акции:
		\[PrS = Q_{prs} \times P_{prs}\]
		\[PrS = 1 \times 10^{6} \times \text{110 руб.} = \textbf{110 млн.руб.}\]
	\end{itemize}
		Остальные статьи берем из баланса (т.е. по балансово	 стоимости).
		\begin{nota}
			Заемный капитал, в данном случае, состоит только из долгосрочных кредитов, т.к. краткосрочные используются не на постоянной основе.
		\end{nota}
		Доля каждой статьи рассчитывается довольно тривиально и это хорошо видно по таблице:
\begin{center}
\begin{tabular}[0.96\textwidth]{|l|p{3.2cm}|p{3.2cm}|l|l|}
\hline
Источники финансирования & Объём финансирования, млн руб. & Доля в структуре капитала компании w & Доходность r & w*r   \\
\hline
Собственный капитал      & 840                            & 0.69                                 & 0.1375       & 0.10  \\
\hline
Привилегированные акции  & 110                            & 0.09                                 & 0.121        & 0.01  \\
\hline
Заёмный капитал          & 260                            & 0.21                                 & 0.1          & 0.02  \\
\hline
\textbf{Итого:}                   & \textbf{1210}                           & \textbf{1}                                    &              &       \\
\hline
\end{tabular}
\end{center}

Для расчета доходности r (или же затрат компании) проведем следующие операции:
\begin{itemize}
	\item \textbf{Собственный капитал (r_{e}):}

	\[r_{e} = \frac{d_{s}\times(1+g)}{P_{0}} + g\]
	\[g = ROE \times RR = 0.4 \times 0.125 = 0.05\]
	\[r_{e} = \frac{15\times(1+0.05)}{180} + 0.05 = \textbf{0.1375}\]
	\item \textbf{Привилегированные акции (r_{prs}):}

	\[r_{prs} = \frac{d_{prs}}{P_{prs}}\]
	\[r_{prs} = \frac{13.31\%}{110}} = \textbf{0.121}\]
	\item \textbf{Заемный капитал (r_{d}):}

	\[260 = \sum_{n=1}^{7}\frac{53.41}{(1+r_{d})^{n}}\]
	\[4.8684 = \sum_{n=1}^{7}\frac{1}{(1+r_{d})}\]
	\[r_{d} = \textbf{0.1}\]
\end{itemize}

Наконец, рассчитываем WACC с поправкой на налог:
\[WACC = [w_{e}\times r_{e}] + [w_{prs}\times r_{prs}] + [w_{d} \times r_{d}] \times (1-Taxrate)\]
\[WACC = \textbf{12.36\%}\]
	\end{solution}
\end{problem}

\begin{problem}
	Упрощенный баланс компании «Технострой», работающей на российском рынке, представлен следующими данными, млн. руб.:

	\begin{center}
	\centering
	\begin{tabular}{|l|l|l|l|}
	\hline
	Основные средства & 626 & Обыкновенные акции              & 150  \\
	\hline
	Оборотные активы  & 327 & Добавочный капитал              & 75   \\
	\hline
	                  &     & Нераспределенная прибыль        & 125  \\
	\hline
	                  &     & Облигации                       & 250  \\
	\hline
	                  &     & Долгосрочный банковский кредит  & 100  \\
	\hline
	                  &     & Краткосрочный банковский кредит & 83   \\
	\hline
	                  &     & Кредиторская задолженность      & 170  \\
	\hline
	Итого             & 953 &                                 & 953  \\
	\hline
	\end{tabular}
\end{center}
Известно, что краткосрочные банковские кредиты используются как постоянный источник финансирования.
\begin{itemize}
	\item краткосрочные кредиты 3 года назад компания привлекала по ставке 12\% годовых, но поскольку ставки в экономике снижаются вслед за ключевой ставкой, то банки предоставляют кредит компании по ставке 10\% годовых;
	\item долгосрочный кредит выдан банком на 5 лет, и будет выплачиваться равными суммами в размере 29,128 млн. руб. ежегодно;
	\item у  компании  выпущено  250000  облигаций,  номинальная  стоимость  которых составляет 1000 руб., ежегодный купон –11,81\%, но рыночная цена облигаций снизилась и сейчас составляет 868 руб., оставшийся срок до погашения –6 лет;
	\item акции компании сегодня продаются по 160 руб. (дивиденд выплачивается один раз в конце года); в обращении находится 2,5 млн. обыкновенных акций;
	\item $\beta$-коэффициент компании с учетом долга равен 1,5; безрисковая ставка оценивается в 4\%, премия за рыночный риск составляет 8\%;
	\item корпорация является стабильно растущей;
	\item балансовая стоимость банковских кредитов совпадает с их рыночной стоимостью. Рассчитайте средневзвешенные затраты на капитал для данной компании.
\end{itemize}
Налог на прибыль корпорации составляет 25\%. Рассчитайте WACC компании.
\begin{solution}
	Сначала рассчитаем рыночную стоимость отдельных источников финансирования и общий объем финансирования.
\begin{itemize}
	\item[(а)] Собственный капитал (по рыночной оценке):
	\[Equity = Q_{s} \times P_{s} + \textit{Добавочный капитал} + \textit{Нераспределенная прибыль}\]
	\[Equity = 2.5 \times 10^{6} \times \text{160 руб.} +  \text{75 млн.руб.} +  \text{125 млн.руб.} = \textbf{600 млн.руб.}\]
	\item[(b)] Облигации:
	\[Bonds = Q_{bnd} \times P_{bnd}\]
	\[Bonds = 2.5 \times 10^{6} \times \text{868 руб.} = \textbf{217 млн.руб.}\]
\end{itemize}
	Остальные статьи берем из баланса (т.е. по балансово	 стоимости).
	Доля каждой статьи рассчитывается довольно тривиально и это хорошо видно по таблице:
\begin{center}
\begin{tabular}[0.88\textwidth]{|p{3.2cm}|p{3.2cm}|p{3.2cm}|l|l|}
\hline
Источники финансирования         & Объём финансирования, млн руб. & Доля в структуре капитала компании w & Доходность r & w*r     \\
\hline
Собственный капитал              & 600                            & 0.6                                  & 0.16         & 0.0960  \\
\hline
Краткосрочные банковские кредиты & 83                             & 0.083                                & 0.1          & 0.0083  \\
\hline
Долгосрочные банковские кредиты  & 100                            & 0.1                                  & 0.14         & 0.0140  \\
\hline
Облигации                        & 217                            & 0.217                                & 0.15         & 0.0325  \\
\hline
\textbf{Итого}:                           & \textbf{1000}                           & \textbf{1}                                    &              &         \\
\hline
\end{tabular}
\end{center}

Для расчета доходности r (или же затрат компании) проведем следующие операции:
\begin{itemize}
	\item \textbf{Затраты на собственный капитал по САРМ (r_{e}):}

	\[r_{e} = r_{f} + \beta \times (r_{m} - r_{f})\]
	\[r_{e} = 0.04 + 1.5 \times 0.08 = \textbf{0.16}\]
	\item \textbf{Облигации (r_{bnd}):}

	\[r_{bnd} = \frac{C + \frac{MV - P_{0}}{n}}{\frac{MV - P_{0}}{2}}\]
	\[r_{bnd} = \frac{118.1 + \frac{1000 - 868}{6}}{\frac{1000 - 868}{2}} = \textbf{0.15}\]
	\item \textbf{Краткосрочный кредит (r_{sc}):}

		Краткосрочный банковский кредит учитываем при расчете   WACC, т.к. он используется как постоянный источник финансирования. По условию $r_{sc} = \textbf{10\%}$.
	\item \textbf{Долгосрочный кредит (r_{d}):}

		\[100 = \sum_{n=1}^{5}\frac{29.128}{(1+r_{d})^{n}}\]
		\[3.4331 = \sum_{n=1}^{5}\frac{1}{(1+r_{d})}\]
		\[r_{d} = \textbf{0.14}\]
\end{itemize}

Наконец, рассчитываем WACC с поправкой на налог:
\[WACC = \left[w_{e}\times r_{e}\right] + \left[w_{bnd}\times r_{bnd}\right]\times (1-Taxrate) + \left[w_{d} \times r_{d}\right] \times (1-Taxrate) + \left[w_{sc} \times r_{sc}\right]\times (1-Taxrate)\]
\[WACC = \textbf{13,71\%}\]
\end{solution}
\end{problem}

\subsection{Задача с расчетом левереджированного бета-коэффициента.}

\begin{problem}
	Корпорация  «Нанотехнология»  занимается  разработкой  транзисторов  нового поколения для использования их в электронике. До текущего года компания финансировала свою деятельность только за счет собственных источников, и ее акционерный капитал оценивался в 300 млн. руб. В этом году компания решила выкупить часть своих акций, чтобы стабилизировать их курс и по возможности в дальнейшем повысить его. Компания выпустила для этого облигации на 120 млн. руб. Известно также, что безрисковая ставка процента  оценивается  в  7,1\%,  уровень  среднерыночной  доходности –17,6\%.  Когда корпорация  была  финансово  независимой,  то  ее $\beta$-коэффициент  оценивался  в  1,2. Появление заемного финансирования в структуре капитала корпорации повысит рыночную стоимость собственного капитала корпорации до 210 млн. руб., рыночная оценка заемного каптала не изменится и составит 120 млн. руб. Компания  платит  налог  в  размере  25\%. Рассчитайте  для  корпорации «Нанотехнология» следующие показатели:
	\begin{itemize}
		\item[\textbf{a:}] $\beta$-коэффициент компании после того, как она стала финансово зависимой;
		\item[\textbf{b:}] требуемую доходность  акционерного  капитала  компании,  имеющей  заемное финансирование;
		\item[\textbf{c:}] средневзвешенные затраты на капитал левереджированной корпорации, если облигации компании с аналогичным рейтингом А имеют доходность 11,6%.
	\end{itemize}
	\begin{solution}
		Известно, что $\beta$-коэффициент финансово независимой компании, т.е. \textbf{$\beta_{U}$=1,2}; еще называют отраслевая $\beta$. Мы можем пересчитать его $\beta$-коэффициент финансово зависимой, или левереджированной, корпорации:
		\[\beta_{L} = \beta_{U}\left[1 + (1-Taxrate)\frac{D}{Eqty}\right]\]
		\[\beta_{L} = 1.2\left[1 + (1-0.25)\frac{120}{210}\right] = \textbf{1.71}\]

		Требуемая доходность собственного капитала финансово зависимой компании составит:
		\[r_{e}^{L} = r_{f} + \beta_{L}\times(r_{m} - r_{f})\]
		\[r_{e}^{L} = 7.1\% + 1.71\times(17.6\% - 7.1\%) = \textbf{25.1\%}\]

		WACC для левереджированной компании:
		\[WACC^{L} = \left[0.251\times \frac{210}{120 + 210}\right] + \left[0.116 \times \frac{120}{120+210} \times (1-0.25)\right] = \textbf{19.13\%}\]
	\end{solution}
\end{problem}

\subsection{Задача на расчет затрат на капитал в случае финансирования нового проекта}
\begin{problem}
	Корпорация  «Энергомир»  участвует  в  конкурсе  по  разработке  проекта использования энергии  ветра  на  тех  территориях  России,  где  нет централизованного энергоснабжения и привозное топливо является дорогим (в частности, на территории крайнего  Севера  и  др.).  У  компании  уже  есть  успешно  внедренные  проекты  по использованию других возобновляемых источников энергии, в частности, энергии морских приливов. Корпорация является успешной, втекущем году ее чистая прибыль составит 825 млн. руб. В развитие бизнеса компания стабильно реинвестирует 80\% чистой прибыли, остальную  часть  выплачивает  в  виде  дивидендов.  Темп  роста  дивидендов  является стабильным и составляет 5\%, а последний выплаченный дивиденд был равен 24 руб. на акцию. Сейчас акции продаются по 250 руб.

	Для финансирования нового проекта корпорации необходимо финансирование в объеме 3 млрд. руб. Реализация проекта должна осуществиться за 5 лет. Компания имеет следующую структуру капитала, которую можно считать оптимальной: 40\% заемного капи-тала; 10\% привилегированных акций; 50\% капитала, состоящего из обыкновенных акций и нераспределенной прибыли.
	\begin{itemize}
		\item Для  финансирования  проекта  компания  планирует  использовать  банковский синдицированный кредит под 13,5\% годовых на 5 лет.
		\item Кроме  того,  компания  выпустит  привилегированные  акции,  которые  будут размещены  по  номинальной  стоимости  120  руб.  за  штуку  с  установленным дивидендом в размере 10 руб., затраты на размещение составят 2\% от номинала акции.
		\item По расчетам финансового директора, компания должна будет осуществить также выпуск новых обыкновенных акций, если не будет достаточно нераспределенной прибыли. Затраты на эмиссию составят 9\% от текущей рыночной цены акций. Предельная ставка налогообложения корпорации 25\%.
	\end{itemize}
	Для корпорации «Энергомир» рассчитайте:
	\begin{itemize}
		\item[\textbf{a:}] величину средневзвешенных затрат на капитал при финансировании нового проекта;
		\item[\textbf{b:}] точку перелома нераспределенной прибыли;
		\item[\textbf{c:}] предположим,  что  в  текущем  году  амортизационные  отчисления  компании «Энергомир» составят 150 млн. руб., а величина отсроченных к выплате налогов —85 млн. руб. Какое влияние это окажет на WACC компании и на график МСС?
	\end{itemize}
	\begin{solution}
		Компания должна финансировать свой проект так, чтобы поддерживать структуру финансирования  (заемный капитал  40\%, привилегированные акции  10\%; собственный капитал  50\%). Поскольку для финансирования проекта в  3 млрд. руб. нераспределенной прибыли текущего года недостаточно, то компании нужно выпустить акции. Поэтому нужно рассчитать доли отдельно для выпуска новых акций и финансирования за счет нераспределенной прибыли: Всего объем финансирования: 3 млрд. руб.

		\begin{itemize}
			\item[\textbf{а)}] Заемный капитал
			\[\text{3 млрд.руб.} \times 0.4 = \textbf{1.2 млрд.руб.}\]
			\item[\textbf{б)}] Привилегированные акции
			\[\text{3 млрд.руб.} \times 0.1 = \textbf{0.3 млрд.руб.}\]
			\item[\textbf{в)}] Нераспределенная прибыль
			\[\text{0.825 млрд.руб.} \times 0.8 = \textbf{0.66 млрд.руб.}\]
			\item[\textbf{г)}] Акции нового выпуска
			\[\text{3 млрд.руб.} \times 0.5 - \text{0.66 млрд.руб.}  = \textbf{0.84 млрд.руб.}\]
			\begin{nota}
				Мы нашли общий объем финансирования за счет собственного капитала, а потом вычли нераспределенную прибыль, поэтому оставшийся собственный капитал будем финансировать за счет выпуска новых обыкновенных акций.
			\end{nota}
		\end{itemize}

		Рассчитаем затраты на отдельные источники финансирования.
		\subparagraph{Затраты на синдицированный кредит} вычислять не нужно, т.к. по условию затраты составляют $r_{d} = 13,5\%$.
		\subparagraph{Затраты на собственный капитал} (на нераспределенную прибыль, т.е. пока не берем эмиссию новых акций)
		\[r_{np} = \frac{Div_{0}\times(1+g)}{P_{0}}+g\]
		\[r_{np} = \frac{24\times(1+0.05)}{250}+0.05 = \textbf{0.1508}\]
		\subparagraph{Затраты на акции нового выпуска} Пусть F  – уровень затрат на размещение обыкновенных акций, выраженный в долях единиц  (в затраты входят затраты выпуска, как затраты на печать, комиссионные инвестиционного банка).
		\[r_{ns} = \frac{Div_{1}}{P_{0}\times(1-F)}+g\]
		\[r_{ns} = \frac{24\times(1+0.05)}{250\times(1-0.09)}+0.05 = \textbf{0.1608}\]
		\subparagraph{Затраты на привилегированные акции}
		\[r_{p} = \frac{D_{p}}{P_{p} - F_{p}}\]
		\[r_{p} = \frac{10}{120 - 2.4} = \textbf{0.085}\]

		\subparagraph{Расчет WACC} При расчете  WACC нужно учитывать, что берем не весь собственный капитал, а отдельно доли нераспределенной прибыли и отдельно долю финансирования за счет новых акций:
		\[WACC = \textbf{12.72\%}\]

		Чистая прибыль текущего года    –    0,825 млрд. руб. При коэффициенте реинвестирования  0,8 нераспределенная прибыль текущего года составляет  0,66 млрд. руб. Поскольку оптимальная доля собственного капитала составляет  50\%, то общий объем финансирования, который компания может привлечь, не выпуская дополнительно новые обыкновенные акции, составит Z. Определим эту величину.
		\[0.66 = Z \times 0.5\]
		\[Z = \textbf{1.32 млрд.руб.}\]
	\end{solution}
\end{problem}

\subsection{Задача  на  расчет WACC при  выкупе  акций  и  изменении  долей  долга  и собственного капитала.}
\begin{problem}
		У корпорации «Атланта» в настоящее время  в обращении находится 800 тыс. обыкновенных акций. Балансовая стоимость акционерного капитала компании равна 72 млн. руб. Сейчас обыкновенные акции продаются по цене, в 2,5 раза выше балансовой стоимости акций. Известно также, что $\beta$-коэффициент акций данной компании равен 1,1. Безрисковая ставка в данный момент составляет 6,1\%, а среднерыночная доходность – 17,1\%. Облигации корпорации имеют рейтинг Аа. Коэффициент долговой нагрузки в рыночных ценах равен 0,4 (отношение долга ко всем активам).

		Корпорация опасается скупки ее акций и планирует выпустить облигации на 30 млн. руб. и на эти средства выкупить часть акций на рынке. Однако это повысит $\beta$-коэффициент до 1,2. Кроме того, рейтинговое агентство понизит рейтинг компании до А. Согласно расчетам, стоимость компании после выкупа не изменится.

		Известно, что в зависимости от рейтинга дополнительная доходность к безрисковой ставке по облигациям составит:
\begin{center}
\begin{tabular}{|l|l|l|l|l|}
\hline
                           & Ааа    & Аа     & А      & Baa     \\
\hline
Спред к безрисковой ставке & 3.80\% & 4.90\% & 5.90\% & 6.50\%  \\
\hline
\end{tabular}
\end{center}
		Ставка налогообложения корпорации – 25\%. Определите для корпорации «Атланта»:
		\begin{itemize}
			\item[\textbf{a:}] средневзвешенные затраты на капитал (WACC) до выкупа акций;
			\item[\textbf{b:}] средневзвешенные затраты на капитал (WACC) после выкупа акций.
		\end{itemize}
	\begin{solution}
		Номинальная стоимость одной акции:
		\[P_{nom} = \frac{72000000}{800000} = \textbf{90 руб.}\]
		По условию рыночная цена в 2.5 раза выше балансовой, значит:
		\[P_{markt} = 90 \times 2.5 = \textbf{225 руб.}\]
		Рассчитаем ключевые параметры (акционерный капитал, общий объем финансирования затраты на собственный капитал, затраты на облигации и WACC) до выкупа акций:
		\[Eqty^{0} = 225 \times 800 000 = \textbf{180 млн.руб.}\]
		\[TFV = \frac{180}{0.6} = \textbf{300 млн.руб.}\]
		\[w_{e}^{0} = \textbf{0.6}\]
		\[r_{e}^{0} = 0.061 + 1.1 \times (0.171 - 0.061) = \textbf{0.1820}\]
		\[r_{d}^{0} = 0.061 + 0.049 = \textbf{0.11}\]
		\[WACC^{0} = \left[0.6 \times 0.1820\right] + \left[0.4 \times 0.11 \right] \times (1-0.25) = \textbf{14.22\%}\]

		В результате выкупа β-коэффициент компании возрос (по условию), поэтому пересчитаем затраты на собственный капитал компании:
		\[r_{e}^{0} = 0.061 + 1.2 \times (0.171 - 0.061) = \textbf{0.1930}\]

		Вследствие выкупа акций изменились рыночные доли, при этом общий объем капитала остался неизменным. Получается, что
		\[Eqty^{1} = 180 - 30 = \textbf{150 млн.руб.}\]
		\[w_{e}^{1} = \textbf{0.5}\]
		Также, за счет понижения рейтинга компании по облигациям с Аа до А, повысится на 1\% и требуемая доходность $r_{d}$. В соответствии с этими данными пересчитаем WACC:
		\[WACC^{1} = \left[0.5 \times 0.1930\right] + \left[0.5 \times 0.12 \right] \times (1-0.25) = \textbf{14.15\%}\]
	\end{solution}
\end{problem}

\begin{problem}
	Крупная корпорация, работающая в химической отрасли, достаточно стабильно развивается, осуществляет множество исследований и постоянно разрабатывает новые виды готовой продукции. В обращении находится 402,5 тыс. акций компании. Показатель чистой прибыли за последние 6 лет показывал следующую динамику:
\begin{center}
\begin{tabular}{|c|c|c|c|c|c|c|}
\hline
                         & 2012  & 2013  & 2014  & 2015  & 2016  & 2017 (exp.)  \\
\hline
Чистая прибыль, тыс.руб. & 20000 & 22000 & 24200 & 26620 & 29282 & 32210        \\
\hline
\end{tabular}
\end{center}

	Ожидается, что сложившийся темп прироста прибыли сохранится и в будущем.

	Корпорация ежегодно реинвестирует в производство 60\% своей прибыли. Акции корпорации продаются в настоящее время по 400 руб.

	Корпорация  имеет  также  заемный  капитал  в  виде  долгосрочного  банковского кредита, по которому платила 15\% годовых, но в текущем году в связи с хорошей кредитной историей и снижением ставок в экономике банк готов предоставить финансирование по  13,5\% годовых. Коэффициент долговой нагрузки для корпорации равен 0,4. Компания платит налог на прибыль по ставке 25\%.
	\begin{itemize}
		\item[\textbf{a:}] Определите темп роста дивиденда на акцию данной корпорации.
		\item[\textbf{b:}] Какой дивиденд на акцию будет выплачен по итогам 2017г.?
		\item[\textbf{c:}] Рассчитайте  средневзвешенные  затраты  на  капитал,  исходя  из  представленных данных.
	\end{itemize}

	\newpage

	\begin{solution}
		Исходя из динамики чистой прибыли компании, можно заключить, что темп роста дивидендов составляет $\textbf{g = 10\%}$.
		Рассчитываем дивиденд для 2017 года:
		\[DPS_{17} = \frac{PR \times (1-RR)}{Q_{s}}\]
		\[DPS_{17} = \frac{32210 \times 0.4}{402.5} = \textbf{32 руб.}\]
		Рассчитаем затраты на собственный капитал компании по модели постоянного роста (при учете роста прибыли и дивидендов в 10\%).

		Поскольку прибыль на конец 2017 года является ожидаемой, еще не получена, то и дивиденд считаем ожидаемым.
		\[r_{e} = \frac{D_{16}\times(1+g)}{P_{0}} + g\]
		\[r_{e} = \frac{32}{400} + 0.1 = \textbf{0.18}\]

		Рассчитаем долгожданный WACC:
		\[WACC = [0.6 \times 0.18] + [0.4 \times 0.135] \times (1-0.25) = \textbf{14.85\%}\]
	\end{solution}
\end{problem}

\section{Операционный и финансовый риск компании}

\begin{problem}
	Компания «Эквалайзер» производит МР3 плееры. Расходы корпорации в текущем году составили, в млн. руб.:
	\begin{itemize}
		\item Заработная плата управляющих --- \textbf{33.5}
		\item Амортизация оборудования, зданий --- \textbf{13}
		\item Расходы на аренду --- \textbf{98.7}
	\end{itemize}
	Цена одного МР3 плеера --- 4500 руб. Затраты на детали и комплектующие --- 1200 руб. на единицу. Согласно прогнозам маркетинговой службы компании, сбыт плееров в текущем году составит 120 000 штук. Известно также, что компания ежегодно выплачивает лизинговые платежи в размере 560 тыс. руб., а также выплачивает по банковскому кредиту в 80 млн. руб. процентные платежи в размере 12,5\% в год. Определить:
	\begin{itemize}
		\item[\textbf{a:}] Степень операционной и финансовой зависимости
		\item[\textbf{b:}] Предположим,  что у компании возросли расходы на аренду на  15  млн. руб. Рассчитайте новый уровень операционной зависимости и прокомментируйте получившийся результат.
		\item[\textbf{c:}] Используя первоначальные данные, рассчитайте эффект финансового рычага для ситуации,  когда долг компании увеличивается за счет выпуска облигаций на  310  млн. руб.  с купонной ставкой 11\% годовых. Прокомментируйте изменение полученного показателя.
	\end{itemize}
	\begin{solution}
		Рассчитаем постоянные издержки: $FC = 33.5 + 13 + 98.7 = \textbf{145.2 млн.руб.}$
		\subparagraph{Эффект операционного рычага}
		\[DOL = \frac{Q(P-V)}{Q(P-V)-FC} = [...] = \textbf{1.58}\]
		\subparagraph{Эффект финансового рычага}
		\[DFL = \frac{Q(P-V)-FC}{Q(P-V)-FC-I-\frac{PD}{1-t}} = [...] = \textbf{1.044}\]
		\subparagraph{Эффект совокупного рычага}
		\[DTL = DOL \times DFL = [...] = \textbf{1.65}\]

		С учетом расходов на аренду эффект операционного рычага составит:
		\[DOL = \frac{396}{250.8-15} = \textbf{1.68}\]

		В пункте (b) присутствуют дополнительные купонные расходы $\Rightarrow$ I возрастает.
		\[I^{'} = 310 \times 0.11 = \textbf{34.1 млн.руб.}\]
		\[DFL^{'} = \frac{250.8}{240.24-34.1} = \textbf{1.217}\]
	\end{solution}
\end{problem}

\begin{problem}
	На данный момент ресторанная сеть «Гурман» выпустила долговых обязательств на 3 млн.  дол.  под  12\%  годовых.  Компания намерена финансироват ьпрограмму расширения стоимостью   4   млн.   дол.   и рассматривает   3   возможных варианта:   эмиссия долговых обязательств под   14\%   годовых,   эмиссия привилегированных акций с выплатой   12\% дивидендов или эмиссия обыкновенных акций по цене 16 дол. за акцию. В настоящее время компания уже выпустила    800000    обыкновенных акций,    кроме того,    уровень ее налогообложения составляет 40\%.
	\begin{itemize}
		\item[\textbf{a:}] Если сейчас EBIT равен  1,5  млн.  дол.,  какую величину составит EPS по трем вариантам при условии, что прибыльность тотчас же не увеличится?
		\item[\textbf{b:}] Рассчитайте точку безразличия для эмиссии облигаций и эмиссией обыкновенныхакций. Постройте графики в координатах «EBIT-EPS» по всем трем вариантам.
		\item[\textbf{c:}] Какой из вариантов предпочли бы вы при данном уровне EBIT? Насколько должен увеличиться EBIT, чтобы другой вариант стал лучшим?
	\end{itemize}
	\begin{solution}

	\end{solution}
\end{problem}

\begin{problem}
	Компания  «Knauff»  выпускает широкий спектр строительных материалов,  которые используются в отделке жилых и промышленных помещений. Компании необходимо финансирование в объеме  10000  млн.  руб.  для осуществления проекта по строительству нового завода. Согласно первому варианту планируется выпустить  500  млн.  обыкновенных акций по 20 руб., а по второму --- выпустить 10 млн. облигаций на 4 года, номиналом по 1000 руб. с 11\% купоном. Известно, что у компании уже есть 500 млн. обыкновенных акций в обращении, а также 10 млн. привилегированных акций, номиналом в 50 руб. и фиксированным дивидендом в 10\%. Текущий уровень операционной прибыли равен 24000 млн. руб. Компания платит налог на прибыль по ставке 40\%. Определите для компании «Knauff»:
	\begin{itemize}
		\item[\textbf{a:}] EPS для каждого варианта финансирования;
		\item[\textbf{b:}] Точку безразличия для финансирования за счет обыкновенных акций и облигаций.
		\item[\textbf{c:}] Какой из вариантов стоит выбрать компании при данном уровне EBIT?
	\end{itemize}
	\begin{solution}

	\end{solution}
\end{problem}

\section{Теории структуры капитала.}

\begin{problem}
	Компании «Тур» и «Вояж» работают в одной сфере —в туристическом бизнесе, и несут одни и те же риски. Рынок характеризуется высокой конкуренцией, представлен крупными устойчивыми компаниями. Инвесторы ожидают по каждой из них прибыль до выплаты процентов и налогов в размере 24 млн. руб. Различие компаний в том, что «Тур» финансирует свои активы только за счет собственного капитала, а «Вояж» имеет 5-летние облигации на 100 млн. руб. с купонной ставкой 10\% годовых и рейтингом АА. Затраты на собственный капитал компании «Тур» равны 12\%. Все предпосылки модели Модильяни—Миллера без учета налогов сохраняются. Используя выводы модели Модильяни—Миллера, определите:
	\begin{itemize}
		\item[\textbf{a:}] Стоимость корпораций;
		\item[\textbf{b:}] Затраты на собственный капитал компании "Вояж"
	\end{itemize}
	\begin{solution}
		Рассчитываем стоимости нелевереджированной и левереджированной компаний:
		\[V_{U}=V_{L}=\frac{EBIT}{r_{e}^{U}}=\frac{24}{0.12} = \textbf{200 млн.руб.}\]
		\begin{nota}
			Основным положением модели Модильяни-Миллера является утверждение, что рыночная стоимость компании (V) не зависит от структуры капитала, то есть от используемой доли заемных средств ($V_{U}=V_{L}$).

			Таким образом, образом рыночная оценка капитала зависит не от его структуры, а от величины операционной прибыли (EBIT) и требуемой ставки доходности на собственный (акционерный) капитал при нулевой доле заемного капитала ($r_{e}^{U}$).
		\end{nota}
		Определяем затраты на собственный капитал левереджированной компании (Вояж):
		\[D_{L} = \textbf{100 млн.руб.}\]
		\[E_{L} = 200 - 100 = \textbf{100 млн.руб.}\]
		\[r_{e}^{L} = r_{e}^{U} + \textit{Премия за риск} = r_{e}^{U} + (r_{e}^{U}-r_{d})\times \frac{D}{E}\]
		\[r_{e}^{L} = 12\% + (12\%-10\%)\times \frac{100}{100} = \textbf{14\%}\]
	\end{solution}
\end{problem}

\begin{problem}
	Корпорации «Магистраль» и «Автороуд», относящиеся к одной группе риска, занимаются строительством дорог и мостов. Активы указанных корпораций генерируют прибыль до выплаты процентов и налогов в размере 125 млн руб. «Магистраль» не прибегает к использованию заемных источников финансирования в отличие от корпорации «Автороуд», у которой есть облигационный займ, рыночная стоимость которого составляет 100 млн.руб. и купонная ставка — 14,5\%. Облигациям этой корпорации присвоен рейтинг ВВВ. Налог на прибыль корпораций составляет 26\%. Акции корпорации «Магистраль» активно продаются на рынке, их требуемая доходность составляет 18,5\%. Все предпосылки модели Модильяни—Миллера с учетом корпоративных налогов сохраняются. Рассчитайте в соответствии с моделью Модильяни—Миллера:
	\begin{itemize}
		\item[\textbf{a:}] Стоимость корпораций;
		\item[\textbf{b:}] Требуемую доходность по акциям компании "Автороуд"
	\end{itemize}
	\begin{solution}
		Рассчитываем стоимость нелевереджированной и левереджированной компаний по модели Модильяни-Миллера с учетом налогов:
		\[V_{U}=\frac{EBIT\times(1-Taxrate)}{r_{e}^{U}}=\frac{125\times(1-0.26)}{0.185} = \textbf{500 млн.руб.}\]
		\[V_{L}= V_{U} + Taxrate \times D_{L} = \textbf{526 млн.руб.}\]

		Рассчитаем требуемую доходность по акциям компании "Автороуд".
		\[E_{L} = V_{L} - D_{L} = \textbf{426 млн.руб.}\]
		\[r_{e}^{L} = r_{e}^{U} + (r_{e}^{U}-r_{d})\times \frac{D}{E}\times(1-T)\]
		\[r_{e}^{L} = 18.5\% + (18.5\%-14.5\%)\times \frac{100}{426}\times(1-0.26) = \textbf{19.1948\%}\]
	\end{solution}
\end{problem}

\begin{problem}
	Корпорации «Техно» и «Платформа» производят высококачественную бытовую технику, относятся к одному классу риска, ожидаемый инвесторами уровень операционной прибыли составляет 455 млн.руб. Корпорация «Техно» финансово независима, $\beta$-коэффициент компании — 0,95, безрисковая ставка составляет 6,5\%, а среднерыночная до-ходность — 15,7\%. Вместе с тем компания «Платформа» имеет заемное финансирование в объеме 700 млн.руб., представленных 10\% облигациями. Налог на прибыль корпораций 24\%, на доходы от владения акциями — 25\%, на доходы от предоставления займов — 30\%. Все предпосылки модели Миллера сохраняются. Рассчитайте в соответствии с данной моделью:
	\begin{itemize}
		\item[\textbf{a:}] стоимость корпорации «Техно»;
		\item[\textbf{b:}] стоимость корпорации «Платформа»;
		\item[\textbf{c:}] выигрыш от использования долгового финансирования;
		\item[\textbf{d:}] предположим, что налоги на личный доход от владения акциями и на доход от владения облигациями равны нулю. Чему в этом случае равен выигрыш от использования заемного финансирования? Отличается ли он от результата предшествующего пункта и почему?
	\end{itemize}
	\begin{solution}
		Рассчитаем стоимость корпорации «Техно» и стоимость корпорации «Платформа» согласно модели Модильяни-Миллера:
		\[r_{e}^{U} = 6.5\% + 0.95 \times(15.7\%-6.5\%) = \textbf{15.24\%}\]
		\[V_{U} = \frac{455\times(1-0.24)(1-0.25)}{0.1524} = \textbf{1702 млн.руб.}\]
		\[V_{L} = V_{U} + \left[1 + \frac{(1-Txr_{pr})\times(1-Txr_{e})}{(1-Txr_{d})}\right]\times D_{L}\]
		\[V_{L} = 1702 + \left[1 + \frac{(1-0.24)\times(1-0.25)}{(1-0.3)}\right]\times 700 = \textbf{1832 млн.руб.}\]

		Выражение в квадратных скобках, умноженное на $D_{L}$ представляет собой выигрыш от использования долгового финансирования. Получается, в данном случае его легко получить, немного изменив формулу $V_{L}$:
		\[TD = V_{L} - V_{U} = 1832 - 1702 = \textbf{130 млн.руб.}\]
		В случае, когда налоги на личный доход от владения акциями и облигациями равны нулю, выигрыш от использования долгового финансирования станет равен:
		\[TD = \left[1-(1-0.24)\right]\times D_{L} = \textbf{168 млн.руб.}\]
	\end{solution}
\end{problem}

\newpage

\begin{problem}
	Когда строительная корпорация «Новый дом» не привлекала заемные финансовые ресурсы, ее рыночная стоимость составляла 946 млн.руб. Сейчас корпорация взяла в крупном банке кредит в объеме 300 млн.руб. Кредит будет использован на выкуп акций, и сумма источников финансирования не изменится. При таком уровне долга, по оценкам аналитиков, приведенная стоимость ожидаемых затрат, связанных с финансовыми затруднениями, составит 35 млн.руб. Кроме того, выдавший кредит банк, опасаясь, что корпорация может пойти на рисковые проекты или привлечь дополнительные долгосрочные займы, разрабатывает сложное кредитное соглашение. Затраты на разработку соглашения и контроль за деятельностью корпорации в отношении их соблюдения оценивается в 47 млн.руб. Приведенная стоимость этих затрат составляет 38 млн.руб. Корпорация платит налог на прибыль в размере 25\%. Определите стоимость корпорации «Новый дом».
	\begin{solution}
		Рассчитываем стоимость левереджированной компании по модели Модильяни-Миллера с учетом налогов и возможных издержек (в данном случае издержек банкротства и агентских издержек):
		\[V_{L} = V_{U} + TD - PV_{bankrupcy} - PV_{agent}\]
		\[V_{L} = 946 + 0.25\times 300 - 35 - 38 = \textbf{948 млн.руб.}\]
	\end{solution}
\end{problem}

\section{Дивидендная политика компаний}
\subsection{Теоретические аспекты выплат}

\begin{problem}
	Компания «Стерео» занимается выпуском домашних музыкальных  театров,  и  в  отчетном  году  получит  прибыль, равную 400 млн. руб. В обращении находится 20 млн. акций компании. Компания является стабильно растущей. В будущем компания  может  либо  продолжать  работу  в  том  же  режиме, получая  аналогичный  уровень  дохода,  либо  реинвестировать часть прибыли на следующих условиях:
\begin{center}
\begin{tabular}{|p{4.4cm}|p{4.4cm}|p{4.4cm}|}
\hline
Доля реинвестируемой прибыли, \% & Достигаемый темп прироста прибыли, \% & Требуемая доходность по акциям компании, \%  \\
\hline
0                                & 0                                     & 10                                           \\
\hline
10                               & 6                                     & 14                                           \\
\hline
20                               & 9                                     & 14                                           \\
\hline
30                               & 11                                    & 18                                           \\
\hline
\end{tabular}
\end{center}
Какая альтернатива более предпочтительна для компании?
\begin{solution}
	Если компания увеличит норму выплат, то цена акций может возрасти, исходя из формулы:
	\begin{equation}\label{eq:stocksprice}
		P_{s} = \frac{D_{1}}{r_{e}-g}
	\end{equation}

	Но, с другой стороны, рост дивиденда может привести к снижению реинвестирования $\Rightarrow$ к снижению темпов роста масштаба деятельности и в дальнейшем, за счет снижения g цена акций может понизиться. Поэтому нужен баланс между текущими дивидендами и будущим ростом, который максимизирует цену акций компании.

	Мы должны найти величину реинвестирования, и следовательно, объем дивидендных выплат, который обеспечит максимальную цену акций.

	Сначала рассчитаем дивиденд для всех вариантов по формуле:
	\[DPS = \frac{PR\times(1-RR)}{Q_{s}}\]
	Затем рассчитаем соответствующую цену акций компании по формуле \eqref{eq:stocksprice}, приведенной выше и занесем все данные в таблицу:
\begin{center}
\begin{tabular}[0.88\textwidth]{|p{4.4cm}|c|c|}
\hline
Доля реинвестируемой прибыли, \% & DPS, руб. & Price, руб.  \\
\hline
0                                & 20  & 200    \\
\hline
10                               & 18  & 225    \\
\hline
20                               & 16  & 320    \\
\hline
30                               & 14  & 200    \\
\hline
\end{tabular}
\end{center}

Итак, из таблицы хорошо видно, что цена акций максимальна при третьем варианте реинвестирования прибыли корпорацией. Значит выбираем его.
\end{solution}
\end{problem}

\begin{problem}
	Рыночная стоимость компании составляет 856 млн. руб., финансирование осуществляется только за счет собственного капитала, который представлен обыкновенными акциями, продающимися на рынке по 23 руб. По итогам текущего года компания планирует направить на дивиденды 184 млн. руб., и, хотя у компании есть денежные средства именно в данном объеме, но она планирует их полностью потратить на инвестиционный проект с чистой приведенной  стоимостью  64  млн.  руб.  Для  выплаты  дивидендов  привлечение  заемного финансирования не предусматривается.

	Согласно теории Модильяни-Миллера, сколько акций компании будет в обращении и какова будет цена одной акции после выплаты дивидендов соответственно?
	\begin{solution}
		Совокупная рыночная стоимость компании будет равна:
		\[V^{'} = V_{0} + NPV = 856 + 64 = \textbf{920 млн.руб.}\]

		На стр. 415 Б-М, в примере сказано, что собственный капитал берется по рыночной оценке и он равен рыночной стоимости обращающихся акций фирмы (цена одной акции, умноженной на число акций вобращении). То же самое сказано и в статье ММ, поскольку цель статьи - выяснить влияние дивидендной политики на курс акций, и финансирование в оригинальной статье берется только за счет собственного капитала. Поэтому мы можем найти количество акций до выплаты дивидендов:
		\[Q_{0} = \frac{V_{0}}{P_{0}} = \frac{920}{23} = \textbf{40 млн.}\]

		Если компания выплачивает дивиденды, то она должна для этого выпустить новые акции. После выплаты дивидендов стоимость акций старых владельцев сократится на величину выплаченных им дивидендов:
		\[S_{1} = V^{'} - Div = 920 - 184 = \textbf{736 млн.руб.}\]
		\[P_{1} = \frac{S_{1}}{Q_{0}} = \frac{736}{40} = \textbf{18.4 руб.}\]

		Согласно теории ММ, выплата дивидендов не повлияет на стоимость компании, поэтому стоимость останется на прежнем уровне, но возрастет количество акций, которое составит:
		\[Q_{1} = \frac{V^{'}}{P_{1}} = \frac{920}{18.4} = \textbf{50 млн.}\]
	\end{solution}
\end{problem}
\subsection{Практические аспекты выплат}

\begin{problem}
	Имеется следующая информация относительно «Clayton Corporation», взятая из балансового отчета компании (долл.):
	\begin{center}
	\begin{tabular}{|l|c|}
	\hline
	Уставный капитал (номинал 5 долл.) & \$500000   \\
	\hline
	Добавочный капитал                 & \$200000   \\
	\hline
	Нераспределенная прибыль           & \$450000   \\
	\hline
	\textbf{Всего собственный капитал}          & \textbf{\$1150000}  \\
	\hline
	\end{tabular}
	\end{center}

Известно, что текущая стоимость одной обыкновенной акции на рынке составляет \$20. Покажите, как изменятся данные, характеризующие собственный капитал компании, если руководство Clayton Corporation:
\begin{itemize}
	\item[\textbf{a:}] срешило  выплатить  10\%-ый  дивиденд  акциями.  Что  можно  сказать  о  новом значении EPS?
	\item[\textbf{b:}] объявило о дроблении акций (сплит) в пропорции 2 к 1;
	\item[\textbf{c:}] приняло решение об обратном сплите 1:2.
\end{itemize}
	\begin{solution}
		При выплате дивидендов в виде акций номинал акции не изменяется, но 	величина нераспределенной прибыли, из которой выплачивается дивиденд, перераспределяется в \textit{Уставный капитал} и \textit{Добавочный капитал}.

		Разница между рыночной ценой и номиналом определяется как \textit{Добавочный капитал сверх номинала}

		Из условия количество акций в обращении до выплата дивидендов акциями: $\frac{\$500000}{\$5} = \textbf{100000 акций}$.

		Поскольку дивиденды выплачиваются в размере 10\% от номинала, то на дивиденды будет выпущено $100000\times0.1 = \textbf{10000 акций}$.

		Тогда общая стоимость дивидендов по рыночной цене составит $10000\times\$20 = \textbf{\$200000}$.

		Теперь покажем, как это отразится на показателях баланса компании:
		\begin{center}
		\begin{tabular}{|l|c|}
		\hline
		Уставный капитал (номинал 5 долл.) & $\$500000 + 10000\times \$5 = \textbf{\$550000}$  \\
		\hline
		Добавочный капитал                 & $\$200000 + (\$20 - \$5)\times 10000 = \textbf{\$350000}$    \\
		\hline
		Нераспределенная прибыль           & $\$450000 - \$20000 =  \textbf{\$250000}$  \\
		\hline
		\textbf{Всего собственный капитал}          & \textbf{\$1150000}  \\
		\hline
		\end{tabular}
		\end{center}

		Рассмотрим ситуацию при дроблении 2 к 1: В этом случае изменится только статья \textit{Уставный капитал}, все остальные статьи будут как в балансе из условия:
		\begin{center}
		\begin{tabular}{|l|c|}
		\hline
		Уставный капитал \textbf{(номинал 2,5 долл., 200 тыс. акций)} & \$500000   \\
		\hline
		Добавочный капитал                 & \$200000   \\
		\hline
		Нераспределенная прибыль           & \$450000   \\
		\hline
		\textbf{Всего собственный капитал}          & \textbf{\$1150000}  \\
		\hline
		\end{tabular}
		\end{center}

		А теперь -- обратный сплит 1 к 2: В этом случае также поменяется только \textit{Уставный капитал}, количество акций уменьшится, а их номинал возрастет.
		\begin{center}
		\begin{tabular}{|l|c|}
		\hline
		Уставный капитал \textbf{(номинал 10 долл., 50 тыс. акций)} & \$500000   \\
		\hline
		Добавочный капитал                 & \$200000   \\
		\hline
		Нераспределенная прибыль           & \$450000   \\
		\hline
		\textbf{Всего собственный капитал}          & \textbf{\$1150000}  \\
		\hline
		\end{tabular}
		\end{center}
	\end{solution}
\end{problem}

\begin{problem}
	Чистая прибыль крупной металлургической корпорации в текущем году составляет 500 млн. руб. В обращении находится 20 млн. обыкновенных акций предприятия и их рыночная цена составляет 80 руб. за акцию. Компания реинвестирует в производство 70\% своей чистой прибыли, а остальное  выплачивает  в  виде  дивидендов  держателям  обыкновенных акций. Таким образом, корпорация могла бы выплатить дивиденды на сумму 150 млн. руб., однако в этом году общее собрание акционеров приняло решение выкупить акции на эту сумму по цене выше рыночной – 100 руб. за акцию. При этом ожидается, что показатель P/E для данной компании после выкупа акций не изменится.

	Определите:
	\begin{itemize}
		\item[\textbf{a:}] значение EPS для данной компании перед выкупом акций;
		\item[\textbf{b:}] значение EPS после выкупа акций;
		\item[\textbf{c:}] доход от прироста капитала на оставшиеся акции компании после выкупа.
	\end{itemize}
	\begin{solution}
		Рассчитываем значение EPS для компании перед выкупом акций:
		\[EPS_{0} = \frac{PR}{Q_{s}} = \frac{500}{20} = \textbf{25 руб.}\]

		При этом предполагается, что показатель P/E будет постоянным:
		\[\frac{P}{E} = \frac{80}{25} = \textbf{3.2}\]

		На выкуп акций потрачено 150 млн.руб., значит по 100 рублей будет выкуплено
		\[\frac{\text{150 млн.руб}}{\text{100 руб.}} = \textbf{1.5 млн. акций}}\]

		Это значит, что количество акций в обращении после выкупа сократится (но чистая прибыль за этот год остается неизменной).
		\[EPS_{1} = \frac{500}{20 - 1.5} = \textbf{27.03 руб.}\]

		Чтобы понять, насколько изменится цена акций после выкупа, применим мультипликатор P/E к новой величине EPS (помним, что этот показатель для компании не изменяется)
		\[P_{1} = \frac{P}{E} \times EPS_{1} = \textbf{86.5 руб.}\]

		Тогда цена акций для акционеров, не продавших акции, повысится, и прибыль на акцию составит
		\[\Delta P = 86.5 - 80 = \textbf{6.5 руб.}\]
	\end{solution}
\end{problem}

\begin{problem}
	Корпорация «Спектр» производит широкий ассортимент ламп, торшеров, бра и  светильников  для  современных  домов  и  квартир.  Компания  развивается  достаточно успешно, ее прибыль постепенно растет (около 4\% в год). В текущем году чистая прибыль составила 300 млн. руб. На рынке обращается 6 млн. обыкновенных акций компании по цене 150 руб. за акцию. «Спектр» планирует 40\% чистой прибыли текущего года направить на выкуп акций. Сейчас руководство компании рассматривает 2 варианта возможного выкупа акций:
	\begin{itemize}
		\item[\textbf{a:}] в первом случае предполагается выкупить обыкновенные акции по рыночной цене;
		\item[\textbf{b:}] второй вариант предусматривает предложение акционерам выкупить у них акции по цене, на 25\% выше рыночной.
	\end{itemize}
	Предполагается, что после выкупа акций для каждого из вариантов показатель P/E для данной компании не изменится. Расcчитайте доход от прироста капитала на оставшиеся после выкупа акции для обоих случаев.
\end{problem}
	\begin{solution}
	Пункт а) решается аналогично задаче выше, а вот в пункте b) также необходимо учесть, по какой цене будут выкупать акции (на 25\% выше).
	\end{solution}

\begin{problem}
	Согласно финансовым прогнозам на ближайшие 6 лет чистая прибыль корпорации «Кондитер» будет получена в следующих объемах:
\begin{center}
\begin{tabular}[0.88\textwidth]{|l|c|c|c|c|c|c|}
\hline
Год                      & 2015 & 2016 & 2017 & 2018 & 2019 & 2020  \\
\hline
Чистая прибыль, млн.руб. & 378  & 456  & 420  & 389  & 496  & 510   \\
\hline
\end{tabular}
\end{center}
	Сейчас в обращении находится 30 млн. обыкновенных акций, которые можно купить на рынке по 25 руб. за акцию. Рассчитайте, какой дивиденд на акцию получат акционеры в период с 2015 по 2020 годы в следующих случаях:
	\begin{itemize}
		\item[\textbf{a:}] корпорация  выплачивает  в  качестве  дивидендов  27\%  своей  чистой  прибыли (округлите до копеек);
		\item[\textbf{b:}] в 2015 году корпорация решила выплачивать в качестве дивидендов стабильную сумму –4 руб. на акцию, а если прибыль превысит 490 млн. руб., то сумма превышения будет выплачена в форме дивидендов в качестве премии;
		\item[\textbf{c:}] установлен дивиденд в размере 36\% от чистой прибыли, и в 2017 году корпорация осуществила дробление акций в пропорции 2:1 (при этом предполагается, что будет меняться только прибыль, а стоимость компании не изменятся);
		\item[\textbf{d:}] на период 2015-2020 гг. корпорация разработала план капиталовложений, которые составят:
	\end{itemize}
	\begin{center}
	\begin{tabular}[0.88\textwidth]{|l|c|c|c|c|c|c|}
	\hline
	Год                        & 2015 & 2016 & 2017 & 2018 & 2019 & 2020  \\
	\hline
	Капиталовложения, млн.руб. & 250  & 230  & 220  & 210  & 200  & 190   \\
	\hline
	\end{tabular}
\end{center}

	В данном варианте предполагается, что дивиденды выплачиваются по остаточному принципу. В  обращении  по-прежнему  находится  30  млн.  акций.

	При  финансировании проектов корпорация стремится поддерживать оптимальную структуру капитала, при которой доля заемных средств составляет 40\%, а собственных – 60\%. При этом вся чистая прибыль, которая не была потрачена на капиталовложения, будет использована на выплату дивидендов на обыкновенные акции.
	\begin{solution}

	\end{solution}
\end{problem}

\section{Управление оборотным капиталом}

\subsection{Изменение  оборотного  капитала:  влияние  на  денежный  поток  и  отличие  от прибыли.}

\begin{problem}
	Предположим, что компания Beta получила за 2016 год выручку \$200 000, а себестоимость составила \$170 000. На начало года запасы равнялись \$12 000, кредиторская задолженность ---- \$11 000, а дебиторская задолженность --- \$15 000. На конец года запасы составили \$21  000,  кредиторская  задолженность --- \$14  000,  а  дебиторская  задолженность --- \$24  000. Рассчитайте прибыль и денежный поток от операций за год.
	\begin{solution}

	\end{solution}
\end{problem}

\subsection{Управление денежными средствами}

\begin{problem}
	Потребность  транспортно-логистической  компании  в  денежных  средствах можно спрогнозировать, и обычно она составляет 37500 руб. в неделю.

	Затраты на продажу ликвидных ценных бумаг или снятие денег со счета равны 2600 руб.

	Альтернативная ставка вложения имеющихся денежных средств компании оценивается в 15\%.

	Определите:
	\begin{itemize}
		\item[\textbf{a:}] оптимальный остаток денежных средств на счете;
		\item[\textbf{b:}] средний остаток денежных средств на счете;
		\item[\textbf{c:}] количество сделок по продаже ликвидных ценных бумаг в год.
	\end{itemize}
	\begin{solution}
		Оптимальный остаток денежных средств на счете рассчитывается по формуле:
		\[C^{*} = \sqrt{\frac{2F\times T}{r}}\]

		В данном случае годовая потребность в денежных средствах, составит
		\[T_{year} = 52\times T_{week} = 52 \times 37500 = \textbf{1950000 руб.}\]

		Тогда оптимальный остаток денежных средств на счете равен:
		\[C^{*} = \sqrt{\frac{2\times 2600\times 1950000}{0.15}} = \textbf{260000 руб.}\]

		Следовательно, компании нужно взять ссуду или продать ценные бумаги на сумму 260000 руб. в случае, если остаток денежных средств равен нулю (для достижения оптимального остатка). При этом средний оптимальный остаток составит \textbf{130000 руб.}, а количество сделок по проддаже ценных ликвидных бумаг в год:
		\[\frac{T}{C^{*}} = \frac{1950000}{260000} = \textbf{7.5 сделок/год}\]

		При этом сделка будет проводиться один раз в $\frac{365}{7.5} = \textbf{48.6 дней}$ (берем целую часть --- 48 дней).
	\end{solution}
\end{problem}

\subsection{Управление запасами}

\begin{problem}
	Компания  «Безопасная  химия»  выпускает широкий  спектр  товаров  бытовой  химии  и  связана  со множеством поставщиков. Расходы компании на снабжение сырьем  для  производства  стиральных  порошков состоят  из следующих статей:
	\begin{itemize}
		\item На размещение заказов – 1200 руб.;
		\item На телефонные переговоры – 1300 руб.;
		\item Командировочные расходы – 16000 руб.;
		\item На приемку партии – 7100 руб.
	\end{itemize}

	Известно  также,  что  затраты  по  хранению  сырья составляют 10\% от стоимости средних запасов. Менеджеры компании  точно  знают,  что  годовая  потребность  в  запасах составляет 1,6 млн. тонн сырья, которое приобретается по 20000 руб.  за  тонну.  Объемы  реализации  компании  равномерно распределены в течение года.

	Определите для компании «Безопасная химия»:
	\begin{itemize}
		\item[\textbf{a:}] какой объем сырья компания должна заказывать каждый раз у своих поставщиков;
		\item[\textbf{b:}] количество размещаемых заказов в год;
		\item[\textbf{c:}] общие годовые затраты по поддержанию запасов компании «Безопасная химия».
	\end{itemize}
	\begin{solution}
		Определяем постоянные затраты по размещению и выполнению заказов:
		\[FC = 1200 + 1300 + 16000 + 7100 = \textbf{25600 руб.}\]
		Потребность компании в сырье S = 1600000 тонн в год.
		\[C = 10\%\]
		\[P = \text{20000 руб.}\]
		\[EOQ = \sqrt \frac{2FC \times S}{C\times P}\]
		\[EOQ = \sqrt \frac{2\times 25600 \times 1600000}{0.1\times 20000} = \textbf{6400 тонн/год}\]
		Теперь находим количество размещаемых заказов в год:
		\[ \frac{S}{EOQ} = \frac{1600000}{6400} = \textbf{250 заказов/год}\]
		Общие годовые затраты по поддержанию запасов:
		\[TIC = C \times P \times \frac{Q}{2} + FC \times \frac{S}{Q}\]
		\[TIC = 0.1 \times 20000 \times \frac{6400}{2} + 25600 \times 250 = \textbf{12.8 млн.руб.}\]
		\begin{nota}
			В данном случае принимаем $Q=EOQ$.
		\end{nota}
	\end{solution}
\end{problem}

\begin{problem}
	Компания «Биос» занимается сложными исследованиями в области разработки и создания лекарственных препаратов и использует дорогостоящие реактивы для молекулярной биологии, которые имеют особые требования по условиям и срокам хранения, в связи с чем затраты по хранению достаточно высоки. Известно, что годовые затраты, руб., по хранению одной единицы сложного реактива состоят из следующих статей:
\begin{center}
\begin{tabular}{|p{11.4cm}|l|}
\hline
Оплата процентов по кредиту, предоставленному банком для финансирования запасов & 10  \\
\hline
Аренда помещения                                                                & 23  \\
\hline
Страховка                                                                       & 8   \\
\hline
Климат-контроль                                                                 & 9   \\
\hline
Охрана, коммунальные услуги                                                     & 12  \\
\hline
Потери от уценки                                                                & 2   \\
\hline
\end{tabular}
\end{center}

Для  проведения  разработок  компания  используют  ежегодно  4096  упаковок  данного реактива. При этом затраты на размещение и выполнение каждого заказа составляют 512 руб. Рассчитайте оптимальную величину партии заказываемого реактива.
	\begin{solution}
		Для расчета оптимальной величины партии заказа обычно используется формула:
		\[EOQ = \sqrt \frac{2FC \times S}{C\times P}\]

		Где P --- цена покупки единицы запасов;

		С --- годовые затраты по хранению запасов, выраженные в процентах.

		Чтобы записать годовые затраты по хранению единицы запасов в денежном выражении, можем использовать выражение $C^{*} = C\times P$.

		Тогда первоначальная формула предстает в виде:
		\[EOQ = \sqrt \frac{2FC \times S}{C^{*}}\]

		$C^{*}$ --- сумма данных в таблице денежных расходов по хранению единицы запасов. $C^{*} = \textbf{64 руб.}$

		Подставим в формулу и получим:
		\[EOQ = \sqrt \frac{2 \times 512 \times 4096}{64} = \textbf{256 шт.}\]
	\end{solution}
\end{problem}

\subsection{Управление дебиторской задолженностью}

\begin{problem}
	Компания  «Витамин»  реализует  фрукты  для производителей соков на условиях 2/10 нетто 30. При этом 60\% заводов используют скидку и оплачивают по счетам на 10-й день, а остальные --- на  40-й  день.  Безнадежной  дебиторской задолженности у компании не возникает благодаря сотрудничеству с постоянными проверенными покупателями. Сейчас выручка от реализации  составляет  2,7  млн  руб.  Однако  в  текущем  году компания собрала большой урожай фруктов и планирует удвоить реализацию и в таком случае выручка составит 5,4 млн руб. Чтобы реализовать  новый  объем  продукции,  компания  планирует смягчить условия кредитования производителей соков до 5/20 нетто 60.  Ожидается,  что  безнадежныедолги  составят  5\%,  а  из оставшихся 95\% потребителей 80\% воспользуются скидкой, а 20\% оплатят продукцию на 60-й день. Доля переменных затрат в объеме реализации компании равна 0,7. Для фи-нансирования дебиторской задолженности компания использует револьверный кредит по ставке  18\%.  Определите,  выгодна  ли  компании  новая  политика  коммерческого кредитования.
	\begin{solution}
		\[DSO_{0} = 0.6 \times 10 + 0.4 \times 40 = \textbf{22 дня}\]
		\[DSO_{N} = 0.8 \times 20 + 0.2 \times 60 = \textbf{28 дней}\]

		Рассчитаем прирост дебиторской задолженности:
		\[\Delta I = \left[(DSO_{N}-DSO_{0})\times \frac{S_{0}}{360}\right]+VC\times \left[DSO_{N}\times \frac{S_{N}-S_{0}}{360}\right]\]
		\[\Delta I = \left[(28-22)\times \frac{2.7}{360}\right]+0.7\times \left[28\times \frac{5.4-2.7}{360}\right] = \textbf{192000 руб.}\]

		Тогда изменение прибыли составит:
		\[\Delta PR = (S_{N} - S_{0})\times(1-VC) - k\Delta I - (B_{N}S_{N} - B_{0}S_{0}) - (D_{N}S_{N}P_{N} - D_{0}S_{0}P_{0})\]
		\begin{align*}
			B_{0} &= \textbf{0} & B_{N} &= \textbf{5\%} \\
			D_{0} &= \textbf{2\%} & D_{N} &= \textbf{5\%} \\
			P_{0} &= \textbf{60\%} & P_{N} &= \textbf{80\%}
		\end{align*}
		\begin{multline*}
			\Delta PR = (5400000 - 2700000)\times(1-0.7) - \\ - 0.18\times 192000 - (0.05 \times 5400000) - \\
			- (0.05 \times 5400000 \times 0.8 - 0.02 \times 2700000 \times 0.6) = \textbf{321840 руб.}
		\end{multline*}

		Политика выгодна, доналоговая прибыль возрастет на \textbf{321840 руб.}
	\end{solution}
\end{problem}
\end{document}
