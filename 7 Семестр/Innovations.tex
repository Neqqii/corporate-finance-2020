\documentclass[11pt]{article}

\usepackage{amsmath, amssymb, amscd, amsthm, amsfonts}
\usepackage{hyperref}
\usepackage{icomma}                  % "Умная" запятая: $0,2$ --- число, $0, 2$ --- перечисление

\usepackage{indentfirst}

%%% Работа с картинками
\usepackage{graphicx}                % Для вставки рисунков
\graphicspath{{img/}}  				% папки с картинками
\setlength\fboxsep{3pt}              % Отступ рамки \fbox{} от рисунка
\setlength\fboxrule{1pt}             % Толщина линий рамки \fbox{}
\usepackage{wrapfig}                 % Обтекание рисунков текстом


%%% Работа с таблицами
\usepackage{array,tabularx,tabulary,booktabs} % Дополнительная работа с таблицами
\usepackage{longtable}                        % Длинные таблицы
\usepackage{multirow}                         % Слияние строк в таблице

%%% Работа с русским языком
\usepackage{cmap}					% поиск в PDF
\usepackage{mathtext} 				% русские буквы в фомулах
\usepackage[T2A]{fontenc}			% кодировка
\usepackage[utf8]{inputenc}			% кодировка исходного текста
\usepackage[english,russian]{babel}	% локализация и переносы

%%% Страница
\usepackage{extsizes} % Возможность сделать 14-й шрифт
\usepackage{geometry} % Простой способ задавать поля
	\geometry{top=25mm}
	\geometry{bottom=25mm}
	\geometry{left=18mm}
	\geometry{right=14mm}
 %
\usepackage{fancyhdr} % Колонтитулы
 	\pagestyle{fancy}
 	\renewcommand{\headrulewidth}{0mm}  % Толщина линейки, отчеркивающей верхний колонтитул
 	\lfoot{Экономика хреноваций}
 	\rfoot{ЭФ МГУ 2k20}
 	\lhead{Сука какое же говно}
 	\chead{}
 	\rhead{Открытые, блять, вопросы}
 	% \cfoot{Нижний в центре} % По умолчанию здесь номер страницы

\usepackage{setspace} % Интерлиньяж
%\onehalfspacing % Интерлиньяж 1.5
%\doublespacing % Интерлиньяж 2
%\singlespacing % Интерлиньяж 1

\usepackage{lastpage} % Узнать, сколько всего страниц в документе.

\usepackage{soul} % Модификаторы начертания
\usepackage{csquotes} % Цитаты

\usepackage{hyperref}
\usepackage[usenames,dvipsnames,svgnames,table,rgb]{xcolor}
\hypersetup{				                % Гиперссылки
    unicode=true,                           % русские буквы в раздела PDF
    pdftitle={Заголовок},                   % Заголовок
    pdfauthor={Автор},                      % Автор
    pdfsubject={Тема},                      % Тема
    pdfcreator={Создатель},                 % Создатель
    pdfproducer={Производитель},            % Производитель
    pdfkeywords={keyword1} {key2} {key3},   % Ключевые слова
    colorlinks=true,        	            % false: ссылки в рамках; true: цветные ссылки
    linkcolor=red,                          % внутренние ссылки
    citecolor=green,                        % на библиографию
    filecolor=magenta,                      % на файлы
    urlcolor=cyan                           % на URL
}

%\renewcommand{\familydefault}{\sfdefault} % Начертание шрифта

\usepackage{multicol} % Несколько колонок

%%% Теоремы
\theoremstyle{plain} % Это стиль по умолчанию, его можно не переопределять.
\newtheorem{theorem}{Теорема}[section]
\newtheorem{proposition}[theorem]{Определение}

\theoremstyle{definition} % "Утверждение"
\newtheorem{corollary}{Утверждение}[theorem]
\newtheorem{problem}{Задача}[section]

\theoremstyle{remark} % "Примечание"
\newtheorem*{nonum}{Примеры}
\newtheorem{nota}{Примечание}


\title{Экономика инноваций}
\author{Открытые вопросы}
\date{2020}


\begin{document}

\maketitle
\section{Сущность и свойства инноваций.
Новшество, нововведение, инновация.
Инновации как продукт и инновации как процесс.}\label{erste}

Следует  понимать,  что  понятие  «инновация»  относится  к  разряду всеобщих  категорий – исключительно  широких  и  структурно  сложных, имеющих много подходов к раскрытию его содержания.

\begin{proposition}[Инновация]
Внедрённое или внедряемое новшество, обеспечивающее повышение эффективности процессов и (или) улучшение качества продукции, востребованное рынком. Является результатом интеллектуальной деятельности человека, его фантазии, творческого процесса, открытий, изобретений и рационализации
\end{proposition}

\begin{proposition}[Инновация]
Введённый в употребление новый или значительно улучшенный продукт (товар, услуга) или процесс, новый метод продаж или новый организационный метод в деловой практике, организации рабочих мест или во внешних связях
\end{proposition}

\subsection{Сущность и свойства инноваций}

Вводя в научный оборот понятие инновации и давая его характеристику, Й.Шумпетер в  своем  труде  «Теория  экономического  развития»  выделил  ставшие  уже  классическими {\bf «пять типичных изменений»}:

\begin{enumerate}
	\item Внедрение нового продукта, с которым потребитель еще не знаком, либо нового уровня качества существующей продукции.
	\item Внедрение   новых методов   производства,которые   либо основываются  на  научных  открытиях,  либо  могут  представлять  собой  новый способ коммерческого использования продукта или сырья.
	\item Открытие нового рынка, на который еще не заходила определенная отрасль производства некоторой страны, вне зависимости от того, существовал этот рынок раньше или нет.
	\item Захват нового источника сырья или полуфабрикатов, опять же вне зависимости от того, существует ли данный источник, либо он только что был создан.
	\item Реализация изменений  в  организациинекой  отрасли,  в  частности, занятие монопольной позиции (например, через создание трастов), или же его утеря.
\end{enumerate}

\begin{displayquote}
	\textit{"Качественные изменения не только техники и технологии, но и организации производства, являющиеся результатом   сознательной   деятельности предпринимателя."}
\end{displayquote}

\subparagraph{Свойства инноваций}

\begin{itemize}
	\item[1^{\circ}:] Научно-техническая новизна (новый/значительно усовершенствованный продукт)
	\item[2^{\circ}:] Производственная применимость
	\item[3^{\circ}:] Коммерческая реализуемость (способность удовлетворить определенные потребности и запросы потребителей)
\end{itemize}

\subsection{Новшество, нововведение, инновация}
В  понимании  сущности  и  свойств  инноваций  очень  важным  является четкое разграничение понятий «новшества» и «инновации».

\begin{proposition}[Новшество (новация)]
	Оформленный результат фундаментальных, прикладных исследований и разработок в какой-либо сфере деятельности (новое знание,   метод,   изобретение)
\end{proposition}

\begin{corollary}
	Внедрение  новшества,   т.е.   достижение практической   применимости   нового   знания   с   целью   удовлетворения определенных  потребностей  и  рыночного  признания  превращает  его  в \textbf{инновацию (нововведение)}.
\end{corollary}

\begin{nota}
	Инновация отличается от новшества (новации) тем, что новый продукт выходит на рынок, принимается рынком и в дальнейшем будет иметь коммерческую  ценность.  Следовательно,  инновация  (нововведение) означает практическое  применениеданного  новшества.
\end{nota}

\subsection{Инновации как продукт и как процесс}

Инновации  можно  структурировать следующим образом:
\begin{itemize}
	\item[(a)] Инновации-продукты (новые продукты и услуги)
	\begin{itemize}
		\item[---] Инновация  как  продукт рассматривается,  как  правило,  в  более  узком  смысле  слова,  в  предметных, сегментированных  областях.
	\end{itemize}
	\item[(b)] Инновации-процессы (на  микроуровне -новые  технологические процессы  и  способы  организации  производства;  на  макроуровне –изменение структуры рынков и создание новых рынков)
	\begin{itemize}
		\item[---] Изменение    состояния какого-либо  процесса  или  комплекса процессов
	\end{itemize}
\end{itemize}

\newpage


\section{Инновационный процесс. Характеристики линейного и интерактивного инновационного процесса.}\label{erste}

На  наш  взгляд,  следует четко и однозначно разграничивать понятия \textit{«инновация»} и \textit{«инновационный процесс»}.

\begin{proposition}[Инновационный процесс]
	Процесс преобразования научного знания в инновацию (от идеи до конечного продукта  и его дальнейшего практического использования).
\end{proposition}



\subsection{Линейный процесс}

Линейная модель подразумевает определенное разделение труда в инновационном процессе: фундаментальные исследования, как правило, проводятся учеными-исследователями в научных лабораториях академических институтов, университетов, крупных корпораций. Прикладные исследования и экспериментальные разработки чаще выполняются в проектных организациях, конструкторских бюро, отделах НИОКР промышленных фирм. Далее в процесс все больше вовлекаются инженеры, конструкторы, технологи, маркетологи, а также персонал системы сбыта.

\subparagraph{Этапы линейного инновационного процесса}

\begin{itemize}
	\item Фундаментальные исследования
	\item Стадия концептуального проектирования
	\item Разработка
	\item Подготовка производства
	\item Пробный маркетинг (маркетинг-тест)
	\item Процесс коммерциализации
\end{itemize}

\subsection{Интерактивный процесс}

Практика показала, что линейная модель инноваций не отражает всю сложность взаимоотношений между участниками процесса. Анализ сложных инноваций показывает, что между различными стадиями процесса существуют как прямые, так и обратные взаимосвязи, часто вообще невозможно указать, когда появляется изобретение (начальная точка всего процесса), так как идеи инноваций возникают и разрабатываются на всех стадиях инновационного процесса, включая производство и маркетинг продукции.

\begin{itemize}
	\item[1^{\circ}:] Существует более одного прямого пути от исследования до коммерциализации инноваций.
	\item[2^{\circ}:] Фундаментальные исследования не рассматриваются как единственный источник инноваций.
	\item[3^{\circ}:] Результаты исследований используются в той или иной форме на всех стадиях инновационного процесса.
	\item[4^{\circ}:] На всех стадиях могут быть петли обратной связи. Например, рыночное тестирование нового продукта может указать на необходимость доработки конструкции. При невозможности разрешить выявленную проблему в рамках существующих знаний возникает необходимость проведения дополнительных НИР.
	\item[5^{\circ}:] Не существует четкой грани между исследовательским и техническим трудом.
\end{itemize}
\newpage

\section{Базисные, улучшающие и псевдоинновации.}\label{erste}

	\indent Важнейший вклад в теорию инноваций внес немецкий ученый Г.Менш,  осуществивший классификацию инноваций на три крупные группы: базисные, улучшающие и псевдоинновации.

\begin{proposition}[Базисные инновации]
	Продукты,  процессы  или  услуги, обладающие либо невиданными ранее свойствами, либо известными, но значительно улучшенными по производительности или по цене свойствами.
\end{proposition}

Эти радикальные инновации создают такие значительные    изменения    в процессах, продуктах или услугах, что приводят к {\bf трансформации} существующих рынков или отраслей или же {\bf создают} новые  рынки  и отрасли, к примеру, сферу электронного бизнеса благодаря Интернет.

\begin{nonum}
	Изобретение антибиотиков, двигателя, фотографии, рентгена, компьютера
\end{nonum}

\vspace*{2.5em}

\begin{proposition}[Улучшающие  инновации]
	Незначительные,   не революционные  изменения,  во  многом  предсказуемые  и  предопределенные существующими знаниями, продуктами, технологиями.
\end{proposition}

Направлены  на  {\bf развитие и модификацию} базисных  инноваций,  они  намного многочисленнее  их,  но  отличаются значительно  меньшей  новизной  и  более  коротким  жизненным  циклом.

\begin{nonum}
	Multi-touch, новые типы 3D-Принтеров, улучшенные модели процессоров
\end{nonum}

\vspace*{2.5em}

\begin{proposition}[Псевдоинновации]
\end{proposition}

\begin{itemize}
		\item[(a)]
		{\it Незначительные технические или внешние изменения в продуктах, оставляющие  неизменными  конструктивное  исполнение  и  не  оказывающие достаточно  заметного  влияния  на  параметры,  свойства,  стоимость  изделия,  а также входящих в него материалов и компонентов;}
		\item[(b)]
		{\it Расширение   номенклатуры   продукции   за   счет   освоения производства  не  выпускавшихся  прежде  на  данном  предприятии,  но уже известных  на  рынке  продуктов,с  целью  удовлетворения  текущего  спроса  и увеличения доходов предприятия.}
\end{itemize}

Псевдоинновации  распространены,  как  правило,  на  заключительной стадии жизненного цикла системы, когда она  уже в основном исчерпала свой потенциал, но всячески сопротивляется замене более прогрессивной системой.

\begin{nonum}
	модель автомобиля «Жигули», модернизация кассетных магнитофонов, когда есть CD и прочие носители информации
\end{nonum}

\newpage

\section{Понятие технологических укладов.}\label{erste}

Базисные   инновации   лежат   в   основе   последовательной   смены технологических укладов. В каждый момент времени совокупность технологий может  быть  представлена технологическими укладами,  обобщающими  цикл общественного  воспроизводства.

\begin{proposition}[Технологический уклад]
	специфическая, создающая новое качество, устойчивая совокупность базисных технологий, в которой  преобладает характерный  принцип, способ  функционирования технологий
\end{proposition}

В  экономической  науке  сложилась  обоснованная многими  исследователями  классификация,  включающая  пять  действующих технологических  укладов,  а  также  формирующийся  перспективный  шестой уклад.

\subparagraph{Технологические уклады:}

\begin{itemize}
	\setlength{\itemindent}{4em}
		\item[\textbf{1-й уклад}] механические системы;
		\item[\textbf{2-й уклад}] технологии с использованием пара;
		\item[\textbf{3-й уклад}] технологии с использованием электричества;
		\item[\textbf{4-й уклад}] технологии, основанные на автоматизации и химизации;
		\item[\textbf{5-й уклад}] биотехнология, компьютеризация и электронизация;
		\item[\textbf{6-й уклад}] нанотехнология, генная инженерия, мультимедийные интерактивные информационные системы.
\end{itemize}

\begin{nota}
	Хотя  периоды  распространения  укладов  в  данной  классификации представлены  как  последовательные,  вдействительности  они  совмещены  и соседствуют  друг  с  другом.  Это  можно  наблюдать  как  на  отдельных хозяйствующих субъектах, например на современных автомобильных заводах, оснащенных   не   только   конвейерными   линиями,   но   и   роботами,   и персональными компьютерами, так и на примере целых национальных хозяйств.
\end{nota}

\newpage

\section{Концепция подрывных инноваций: продукты и рынки.}\label{erste}

Скорость процессов инновационного развития в  компаниях  часто  превышает  способность  потребителей  воспринимать  все
новые доработки предлагаемого продукта или услуги. В результате продукция этих компаний становится слишком сложной, слишком дорогой и «мудреной» для  адекватного  восприятия  их  широкой  потребительской  аудиторией.

\vspace{1em}
Следование компаниями данному пути «устойчивого инновационного развития» объясняется их восприятием этого пути в качестве ключевого методауспеха на рынке:  устанавливая  максимальные  цены  для  наиболее  требовательной потребительской аудитории компания получает наивысшую прибыль.

\vspace{1em}
Тем  не  менее,  таким  образом,  компании  открывают  путь  \textbf{«подрывным инновациям»}  (англ. disruptive innovations)  нанаиболее массовый,  нижний сегмент рынка.  \textbf{Подрывная инновация} дает доступной группе потребителей, ранее не имевшей доступа к продукту или услуге, в силу слишком высокой цены или слишком высокой сложности пользованияими.

\subparagraph{Характеристики подрывных бизнесов}

\begin{itemize}
		\item более низкая валовая прибыль,
		\item более узкие целевые рыночные сегменты,
		\item более  простые  продукты  и  услуги,  которые  выглядят  не  столько привлекательно  по  сравнению    с  уже  существующими  решениями  (в рамках традиционных оценочных метрик).
\end{itemize}

\begin{longtable}{|p{200pt}|p{200pt}|}

	\caption{Примеры подрывных инноваций}
	\\
	\hline
	\textbf{Подрывная инновация} & \textbf{«Взорванные» рынки}
	\\
	\hline
	\endfirsthead
	Миникомпьютер & Мэйнфрейм (ЭВМ) \\
	\hline
	Принтер & Печатная машинка \\
	\hline
	Цифровая фотография & Химическая фотография \\
	\hline
	Пароход & Парусное судно \\
	\hline
	Мобильный телефон & Пейджер \\
	\hline
	GPS-навигатор & Карты и план местности \\
	\hline
\end{longtable}

\newpage

\section{Роль предпринимателя в инновационном процессе по Й. Шумпетеру. Предприниматели-инноваторы и предприниматели-консерваторы.}\label{erste}

Й.Шумпетер  существенно  развил  теорию  предпринимательства,  уделяя значительное  место  роли  предпринимателя  в  инновационном  процессе. Он впервые  в  теории  экономического  анализа  дифференцировал  экономических субъектов (предпринимателей) на два вида:

\begin{itemize}
	\item[(a)] \textbf{инноваторов}, проектирующих,  разрабатывающих  и  внедряющих новые технологии, продукты и рынки, создающих новые или модернизирующих старые фирмы, влияющих на изменение в институциональной структуре;
	\item[(b)] \textbf{консерваторов}, эксплуатирующих    наличные    технологии, производящих  старые виды  продукции,  действующих  в  рамках  сложившихся фирм, стремящихся к неизменности институтов.
\end{itemize}

В соответствии со взглядами Й.Шумпетера, предприниматель-инноватор является связующим  звеном между изобретением  и  нововведением, а  его деятельность способствует качественному изменению развития экономики. Функциональная  роль  инноватора-предпринимателя  в  экономике  сводится  к нарушению  равновесия,  созданию  неравновесного  состояния  на  рынках вследствие  инноваций,  что  и  приносит  ему,  помимо  предпринимательской прибыли,  дополнительные  сверх-доходы,  получившие  в  современной  науке название «инновационной ренты» и «инновационной квазиренты».

\section{Десять групп технологий для будущего промышленного производства.}\label{erste}

\newpage

\section{Основные элементы национальной инновационной системы.}\label{erste}

В  основе  разработки  концепции  национальных  инновационных  систем были положены следующие методологические принципы:
\begin{itemize}
	\item Идеи Й.Шумпетера:   инновации   и   научныеразработки --– основа конкурентоспособности  корпораций;  роль  новатора-предпринимателя  в коммерциализации научных разработок.
	\item Идеи Ф.Хайека: особая  роль  знания  в  экономическом  развитии («экономика знаний», обучающаяся «креативная» корпорация).
	\item Идеи Д.Норта: роль институциональной среды --- создание разветвленных формальных  отношений  и  механизмов  обеспечивает  эффективность рынков.
\end{itemize}

\subsection{Национальная Инновационная Система}
	\subsubsection{Научно-производственная среда}
	Совокупность организаций частного и государственного секторов экономики по производству и коммерческой реализации научных знаний и технологий в пределах национальных границ
		\begin{itemize}
			\item университеты и академии, государственные лаборатории;
			\item крупные, средние и малые компании
			\item организации инновационной инфраструктуры (технопарки, бизнес-инкубаторы и др.);
			\item финансовые структуры;
			\item консалтинговые структуры.
		\end{itemize}
	\subsubsection{Институциональная среда}
	Комплекс институтов правового,
	финансового и социального
	характера.

	\vspace{1em}
	Совокупность
	законодательных актов, норм и правил, ведомственных инструкций,
	обеспечивающих взаимодействие субъектов инновационной деятельности с другими сегментами национальной экономики и имеющие прочные национальные корни, традиции, политические и культурные особенности.

\newpage

\section{Основные подходы к оценке рынка инновационной компании. Понятия PAM, TAM, SAM, SOM.}\label{erste}

Эти непонятные, на первый взгляд, аббревиатуры дают нам понять только одно – объем рынка сбыта вашего продукта или услуги.
Они расшифровываются так:
\begin{enumerate}
	\item \textbf{TAM (общий объём целевого рынка)} --– дает понять, сколько клиентов на целевом рынке нуждаются (именно нуждаются, не обязательно могут себе это позволить!) в продуктах или услугах, находящихся в той же категории продуктов/услуг, которые продаете вы.
	\begin{displayquote}
		\textit{"Для кого из потребителей может быть интересен (необходим) ваш продукт и в каком объеме?"}
	\end{displayquote}
	\begin{nonum}
		Вы продаете сайты для бизнеса по всей России. Предположим, что в России 5 млн компаний (ИП и Общества), и 50\% из них нужен сайт. Тогда объем ТАМ составит 2,5 млн сайтов. Если вы делаете сайты по 30 тыс. руб. общий объем целевого рынка составит 75 млрд руб.
	\end{nonum}
	\item \textbf{SAM (доступный объем рынка)} --- клиентский сегмент или объем рынка (доля от ТАМ), в рамках которых потребитель готов купить продукты или услуги – такие же, как предоставляет ваш бизнес.
	\begin{displayquote}
		\textit{"Кто из потребителей и в каком объеме может купить ваш продукт?"}
	\end{displayquote}
	\begin{nonum}
		Вы продаете сайты для бизнеса, которые занимаются розничной продажей чего-либо. Предположим, что таких компаний 800 тыс. Сайт нужен для 80\% таких компаний, т.е. для 640 тыс. компаний – это и есть SAM
	\end{nonum}
	\item \textbf{SOM (реально достижимый объем рынка)} – это объем рынка (доля от SAM), который ваша компания намерена и способна занять, учитывая его стратегию развития и действия конкурентов.}
	\begin{displayquote}
		\textit{"Кто конкретно и в каком объеме будет покупать именно вашу продукцию?"}
	\end{displayquote}
	\begin{nonum}
		Вы продаете сайты для бизнеса, и у вас в компании работает 10 дизайнеров и 10 программистов. Это позволяет вам выпускать, предположим, по 100 сайтов в месяц. То есть объем реально достижимого объема рынка (SOM) – 1200 сайтов в год.
	\end{nonum}
	\item \textbf{PAM (потенциальный объем рынка)} --– это глобальный рынок, не ограниченный географией или другими факторами.
	\begin{nonum}
		В нашем случае это будет весь рынок веб-разработки.
	\end{nonum}
\end{enumerate}

\newpage

\section{Концепция Lean Startup.}\label{erste}
Несмотря на обоснованные расчеты, серьезные бизнес-планы, продуманные бизнес-модели, крупные инвестиции, большинство стартапов терпят крах.

\vspace{1em}
Почему так происходит? Эрик Рис, автор методики «бережливого стартапа» уверен в том, что традиционный подход к развитию бизнеса не применим к стартапам.

\vspace{1em}
Стартап действует в условиях чрезвычайной неопределенности, и это нужно учитывать при его запуске. На первоначальном этапе развития стартапу необходимо оставаться гибким, чтобы учиться на ошибках и максимально быстро проверять гипотезы основателей, а значит, необходимо избегать крупных вливаний и затрат. Именно такой подход лежит в основе метода Lean, цель которого помочь предпринимателям повысить шансы стартапа на успех.

\begin{proposition}[Lean Startup]
	Концепция создания компаний, разработки и выведения на рынок новых продуктов и услуг, основанная на таких понятиях, как научный подход к менеджменту стартапов, подтвержденное обучение, проведение экспериментов, итеративный выпуск продуктов для сокращения цикла разработки, измерение прогресса, и получение ценной обратной связи от клиентов.
\end{proposition}

Используя этот подход, компании могут проектировать продукты и услуги, которые бы соответствовали ожиданиям и потребностям клиентов без необходимости большого объёма первичного финансирования или затратных продуктовых запусков.

\subparagraph{Ключевые принципы:}
\begin{itemize}
	\item \textbf{Предприниматели есть повсюду.}

	Эрик Рис называет предпринимателем любого, у кого есть стартап. А стартап — это «предприятие, цель которого — разработка новых товаров и услуг в условиях чрезвычайной неопределенности. Это значит, что подход „экономичный стартап“ можно применять в компаниях любого размера, даже на очень крупных предприятиях, в любом секторе и в любой отрасли».
	\item \textbf{Предпринимательство — это менеджмент.}

	Стартапу нужен менеджмент нового типа, который будет подходить к условиям чрезвычайной неопределенности. Эрик Рис уверен, что любой современной компании, развитие которой зависит от инноваций, нужна должность «предприниматель».
	\item \textbf{Подтверждение фактами.}

	Задача стартапа — не только производить товары и зарабатывать деньги. Стартапу необходимо непрерывное обучение с применением научного подхода и проверки гипотез опытным путем.
	\item \textbf{Цикл «создать-оценить-научиться».}

	Сначала создать минимально рабочую версию продукта, оценить реакцию потребителей, а потом решить, продолжать идти выбранным курсом или изменить направление.
	\item \textbf{Учет инноваций.}

	Это то, что обычно называют скучными подробностями. Но учет инноваций необходим для улучшения результатов работы стартапа. Учет инноваций — это система критериев и показателей, которые помогают оценить успех (или неудачу) действий стартапа.
\end{itemize}

\newpage

\section{Концепция HADI-циклов и их применение.}\label{erste}

\begin{proposition}[Методология HADI]
	Алгоритм действий, помогающий выстроить работу по тестированию гипотез и их влияния на бизнес.
\end{proposition}

\begin{wrapfigure}{r}{0.3248\linewidth}
	\includegraphics[width=\linewidth]{scale_1200.jpg}
	\caption{HADI-Цикол}
\end{wrapfigure}
Суть HADI очень проста. Почти любое действие оказывает влияние на какую-то определенную метрику. Если изменения «прикрутить» к показателям заранее (сформулировать гипотезы), то весь процесс изменений будет контролируемым. Словом, вы будете понимать, как ваши действия повлияли на результат, и сможете быстро тестировать идеи, отбрасывая нерабочие.

\subparagraph{Когда и где надо внедрять HADI?}

HADI подходит для быстрой проверки «понятных» гипотез, например мы ищем идеальную формулировку для заголовка, самый выгодный баннер для РСЯ, дешевые аудитории для таргетинга на FB и т.д. Важно учитывать, что правильность гипотезы обязательно можно проверить числами и их изменениями.

\subparagraph{Как работать HADI циклами?}

Традиционно в стартапах за цикл берется одна неделя, но вы можете корректировать как в большую, так и в меньшую сторону, в зависимости от условий задачи. Помните про статистическую значимость своих экспериментов, это важно при работе с данными.

\subparagraph{HADI-цикл:}

\begin{enumerate}
	\item \textbf{Hypothesis} Записываем гипотезы, то есть изменения (ваши действия), которые могут улучшить показатели вашего бизнеса.
	\begin{nonum}
		«Изменение текста на первом экране увеличит конверсию в регистрацию на лендинге на 5\%».
	\end{nonum}
	В первую очередь, нужно генерировать гипотезы по наиболее важным для вас показателям. Это значит, что если есть проблема с конверсией в регистрацию, то необходимо генерировать те идеи, которые напрямую повлияют на этот показатель, а улучшения для «корзины» на время отложить.
	\item \textbf{Action} В начале каждого цикла берем несколько гипотез и приступаем к реализации.
	\begin{nonum}
		Переписываем заголовки и добавляем туда выгоду для клиента.
	\end{nonum}
	Секрет успеха – делать все быстро. Если изменение принесет пользу, улучшить его и масштабировать будет несложно.
	\item \textbf{Data} Начинаем сбор данных о показателях, на которые влияет изменение.
	\begin{nonum}
		В нашем примере это конверсия в регистрацию.
	\end{nonum}
	Чтобы показатели отражали реальные изменения, количество посетителей за время тестирования должно быть репрезентативным.
	\item \textbf{Insights} Анализируем, сработала ли гипотеза. Если да, то планируем ее улучшение, масштабируем.
\end{enumerate}

\newpage

\section{Модель SPACE.}\label{erste}

\begin{wrapfigure}{r}{0.3248\linewidth}
	\includegraphics[width=160pt, height=160pt]{space.jpg}
	\caption{Модель S.P.A.C.E.}
\end{wrapfigure}
Рассмотрите вот эту картинку. На ней есть несколько окружностей (в дальнейшем — орбит), которые мы рассмотрим как промежуточные точки для понимания концепции.

\subparagraph{Суть модели:}

5 основных понятий — Supplier, Product, Average, Customer, Evaluation — делят круг на сектора. Это и есть S.P.A.C.E (пространство), в котором летает бизнес. Если ваши точки движения (точки позиционирования) находятся на одной орбите, значит полет и последующий разгон будут успешными. В противном случае вас будет штормить, что в большинстве случаев приводит к потере управления.

\vspace{1em}
Естественно, выход на внешнюю орбиту — это массовый рынок, быстрая оборачиваемость и простая масштабируемость. Но выходить на внешнюю орбиту не обязательно. Главное, чтобы вы находились на одной орбите (на одной линии). Расставьте точки на круге и посмотрите, что у вас получится.

\subparagraph{Элементы модели}

\begin{itemize}
	\item \textbf{Supplier} --- это вы как поставщик продукта.

	Ближе к центру — типаж поставщика «хирург» (прежде чем купить, покупатель долго готовится, но много платит, специалистов мало и т.д.). В середине — терапевт (покупатель пришел, посоветовался, попробовал, наблюдается, полегчало). На внешней орбите — аптека (пришел, ознакомился с краткой инструкцией, купил, принял лекарство, полегчало).
	\item \textbf{Product} --- это ваш продукт. К чему он ближе?

	В центре круга — сложносоставной продукт. Для простоты понимания представьте кухню по индивидуальному заказу (каждый проект нестандартный, заказчик принимает решение долго, реализация происходит в несколько этапов). В середине сектора — смартфоны (купил, настроил, изучил инструкцию, пользуешься). На внешней орбите — купил продукт, поставил и получаешь профит.
	\item \textbf{Average} --- средняя стоимость продукта. От самой высокой в центре круга до доступной большинству потребителей на внешней орбите.
	\item \textbf{Customer} --- количество потенциальных покупателей.

	Если их ограниченное количество, то ваша точка ускорения находится ближе к центру. Если ваш продукт нужен широкому кругу потребителей (массовый рынок), то вы ближе к внешней орбите.
	\item \textbf{Evaluation} --- принятие решения о покупке.

	Решение в центральной части сектора принимается коллегиально, часто в несколько промежуточных этапов. В середине сектора решения принимаются по совету с друзьями или коллегами. На внешней орбите решения принимаются спонтанно: захотел — купил.
\end{itemize}

\newpage

\section{Понятие юнит-экономики.}\label{erste}

Рост и успешность бизнеса зависят от решений, которые принимает руководитель. Чтобы не ошибиться, нужно думать на шаг вперед (а лучше — несколько шагов). Юнит-экономика — один из способов сделать обоснованные выводы о возможном успехе или неудаче. С помощью определенных математических расчетов вы можете узнать, куда движется ваша компания — разоряется или растет, и принять правильные управленческие решения.

\begin{proposition}[Юнит-экономика]
	метод экономического моделирования, используемый для определения прибыльности бизнес-модели, путем оценки прибыльности единицы товара или одного клиента.
\end{proposition}

\begin{itemize}
	\item[---] Как правило, применяется для оценки прибыльности бизнес-идеи стартапа.
	\item[---] Бизнес может быть успешным только если отдельная единица товара или услуги будет прибыльной.
\end{itemize}

\subparagraph{Суть метода:}
Юнит-экономика позволяет увидеть, сколько вы зарабатываете с потока клиентов — поток состоит из юнитов, каждый из которых приносит определенную прибыль (или нет). Если вычислить, сколько приносит каждый юнит и какие расходы при этом несет компания, можно рассчитать, какую прибыль вы получите с определенного потока. По результатам расчета становится ясно, стоит ли масштабировать бизнес, привлекать инвесторов, увеличивать поток или маржинальность сделки.

\begin{nota}
	Важно, что юнитом может называться не только клиент, который заплатил (клиент — принятое определение юнита в первую очередь для SaaS-проектов). Так, например, в мобильных приложениях и играх это будет новый пользователь. А для интернет-издания или сервиса — подписчик (рассылки, демо-версии продукта). Также в качестве юнита можно рассматривать единицу товара.
\end{nota}

\subparagraph{Как рассчитать юнит-экономику?}
Итак, цель ваших расчетов — выяснить, приносит ли проект прибыль. Это определяется соотношением CPA (Cost per Acquisition) — стоимостью привлечения пользователя и LTV (Lifetime Value) — пожизненной ценностью клиента, то есть тем, сколько денег принес вам клиент за все время взаимодействия.

Если расходы на привлечение превышают прибыль, то бизнес убыточен и не сможет приносить деньги владельцам. Если LTV превышает CPA, значит вы в плюсе. Однако здесь важна разница между этими показателями, потому что именно с нее вы будете содержать офис, сотрудников и нести иные расходы. Считается, что для того, чтобы бизнес можно было масштабировать, прибыль должна превышать затраты на привлечение клиента хотя бы в три раза.

\begin{nota}
	Однако есть исключение. Это инвестиционная модель бизнеса, когда основной целью является захват доли рынка. Пример — Uber и «Яндекс.Такси», у которых юнит-экономика убыточна на данный момент, но за счет большой доли рынка они смогут выровнять ее позже.
\end{nota}

\newpage

\section{Гибкая и каскадная модели разработки продуктов.}\label{erste}

Способ разработки выбирается исходя из задач бизнеса, объема работ, времени и бюджета. Наиболее популярными считаются \textbf{Waterfall} - каскадная, и \textbf{Agile} - гибкая. Отметим, что неправильный выбор методики, может отрицательно сказаться на вашем проекте, поэтому учитывайте особенности каждой из них.

\subsection{Гибкая модель разработки}

\begin{proposition}[Agile]
	Система идей и принципов «гибкого» управления проектами, на основе которых разработаны популярные методы Scrum, Kanban и другие. Ключевой принцип — разработка через короткие итерации (циклы), в конце каждого из которых заказчик (пользователь) получает рабочий код или продукт.
\end{proposition}

Гибкие методологии строятся на принципе итераций. Создание нового продукта делится на несколько циклов от одной недели до месяца. В зависимости от особенностей проекта, временные рамки оговариваются отдельно. Каждый цикл представляет собой завершенный мини-проект, в котором есть этапы анализа, планирования, тестирования и реализации. В итоге клиент получает продукт, который, при необходимости, корректируется.

\subparagraph{Главные принципы:}
\begin{itemize}
	\item Эффективное взаимодействие в команде важнее процессов и технологий. Цель --- создание качественного проекта.
	\item Внести необходимые изменения можно в любом из циклов разработки.
	\item Лучший способ получения обратной связи с заказчиком и коллегами --- личное общение.
	\item Создаваемый продукт обновляется в конце каждого цикла или один раз в несколько месяцев.
	\item Готовность к изменениям в процессе разработки важнее, чем беспрекословное следование изначальному плану.
\end{itemize}

\subsection{Каскадная модель разработки}

\begin{proposition}[Waterfall]
	Методика управления проектами, которая подразумевает последовательный переход с одного этапа на другой без пропусков и возвращений на предыдущие стадии.
\end{proposition}

Выбирая данную модель для своего проекта, необходимо понимать, что конечный продукт будет иметь недочеты. Предусмотреть все на этапе анализа и планирования просто невозможно, в процессе разработки могут появится новые требования. Однако, в Waterfall сделать правки в течении проекта невозможно также, как и вернуться на шаг назад. Классический подход представляет из себя каскадную модель, которая базируется на последовательном создании проекта, разбитого на циклы.

\newpage

\section{Понятие интеллектуальной собственности и ее виды в соответствии с российским законодательством.}\label{erste}

\begin{proposition}[Интеллектуальная собственность]
	Совокупность личных неимущественных и имущественных прав (интеллектуальные права) на результаты интеллектуальной деятельности, принадлежащих авторам, их наследникам и иным юридическим и физическим лицам согласно закону или договору.
\end{proposition}

Все объекты, подпадающие под определение интеллектуальной собственности, объединяет ряд признаков:
\begin{itemize}
	\item \textit{Нематериальность} --- эти объекты могут быть выражены в той или иной материальной форме, но защищается и признается интеллектуальной собственностью не форма, а сам объект;
	\item \textit{Новизна} --- каждый вновь создаваемый объект интеллектуальной собственности является уникальным, не похожим на все, что было создано ранее - даже если это переработанное или составное произведение;
	\item \textit{Искусственность создания} --- объект интеллектуальной собственности всегда создается человеком, природные объекты, возникшие сами по себе, интеллектуальной собственностью быть не могут.
\end{itemize}

\subparagraph{Виды интеллектуальной собственности:}
\begin{itemize}
	\begin{multicols}{2}
		\item произведения науки, литературы и искусства;
		\item программы для ЭВМ;
		\item базы данных;
		\item исполнения;
		\item фонограммы;
		\item вещание организаций;
		\item изобретения;
		\item полезные модели;
		\item промышленные образцы;
		\item селекционные достижения;
		\item топологии интегральных микросхем;
		\item секреты производства (ноу-хау);
		\item фирменные наименования;
		\item товарные знаки и знаки обслуживания;
		\item наименования мест происхождения товаров;
		\item коммерческие обозначения.
	\end{multicols}
\end{itemize}

\newpage

\section{Интеллектуальные права и их характеристики.}\label{erste}

Интеллектуальные права --- это законодательно установленная возможность лица распоряжаться данного вида собственностью по своему усмотрению. В том числе допускается передача права на использование третьим лицам либо, напротив, применение в их отношении различных запретов. Данная сфера регулируется Гражданским кодексом Российской Федерации.

\subparagraph{Виды интеллектуальных прав}
\begin{enumerate}
	\item \textbf{Авторское право:}

	Авторским правом регулируются отношения, возникающие в связи с созданием и использованием произведений науки, литературы и искусства. В основе авторского права лежит понятие «произведения», означающее оригинальный результат творческой деятельности, существующий в какой-либо объективной форме. Именно эта объективная форма выражения является предметом охраны в авторском праве. Авторское право не распространяется на идеи, методы, процессы, системы, способы, концепции, принципы, открытия, факты.
	\item \textbf{Смежные права:}

	Группа исключительных прав, созданная во второй половине XX- начале XXI веков, по образцу авторского права, для видов деятельности, которые являются недостаточно творческими для того, чтобы на их результаты можно было распространить авторское право. Содержание смежных прав существенно отличается в разных странах. Наиболее распространенными примерами являются исключительное право музыкантов-исполнителей, изготовителей фонограмм, организаций эфирного вещания.
	\item \textbf{Патентное право} --- система правовых норм, которыми определяется порядок охраны изобретений, полезных моделей, промышленных образцов (часто эти три объекта объединяют под единым названием — «промышленная собственность») и селекционных достижений путём выдачи патентов.
	\item \textbf{Права на средства индивидуализации:}

	Группа объектов интеллектуальной собственности, права на которые можно объединить в один правовой институт охраны маркетинговых обозначений. Включает в себя такие понятия, как: товарный знак, фирменное наименование, наименование места происхождения товара.
	\item \textbf{Право на секреты производства (Ноу-хау)}

	\textit{Секреты производства (Ноу-хау)} — это сведения любого характера (оригинальные технологии, знания, умения и т. п.), которые охраняются режимом коммерческой тайны и могут быть предметом купли-продажи или использоваться для достижения конкурентного преимущества над другими субъектами предпринимательской деятельности.
	\item \textbf{Права на нетрадиционные объекты интеллектуальной собственности}
	\begin{nonum}
		Среди прочего: производственные секреты, достижения селекционеров и т.д.
	\end{nonum}
\end{enumerate}
\newpage

\section{Значение интеллектуальной собственности для развития процессов коммерциализации нововведений.}\label{erste}

Эффективное регулирование отношений собственности является основным средством формирования рыночных отношений в промышленности и в конечном итоге определяет результативность инновационной деятельности предприятия.

\vspace{1em}
Коммерциализация объекта интеллектуальной собственности, как инновационного продукта, осуществляется либо через использование прав в производстве инновационной продукции, либо через передачу прав на них. Передать права на ОИС можно путем:
\begin{itemize}
	\item передачи (уступки) всех имущественных прав другому лицу;
	\item внесения прав в уставный капитал предприятия;
	\item передачи прав пользования другому юридическому или физическому лицу.
\end{itemize}

Существует ряд способов коммерциализации интеллектуальной собственности. Это инжиниринг, промышленная кооперация, передача технологий в рамках совместных предприятий, техническая помощь, франчайзинг, лизинг.

\subparagraph{Значение интеллектуальной собственности:}
\begin{itemize}
	\item[---] Один из решающих факторов создания стоимости
	\item[---] Приносит значительные экономические выгоды
	\item[---] Формирует долгосрочные конкурентные преимущества компании
	\item[---] Увеличивает капитализацию
	\item[---] Защищает бизнес от конкурентов
	\item[---] Открывает возможность получения лицензионного дохода
	\item[---] Позволяет контролировать сферы производства и сбыта
	\item[---] Создает стандарты
\end{itemize}

\end{proposition}
\newpage

\section{«Патентный троллинг».}\label{erste}

\begin{proposition}[Патентный тролль]
	физическое или юридическое лицо, специализирующееся на предъявлении патентных исков.
\end{proposition}

Традиционный алгоритм действий таков: купить спящий, т.е. неиспользуемый, патент, затем подождать, пока кто-то разработает похожую технологию, после чего подать иск о нарушении патентных прав. При этом тролли стараются заключить как можно больше досудебных соглашений, пусть даже на небольшие суммы.

Дело в том, что патентные тяжбы, зачастую, стоят очень дорого, и даже если жертва выиграет, для неё дешевле будет отдать деньги троллю, чем участвовать в суде.

Против развязавших патентную войну конкурентов часто помогают встречные патентные иски (в первую очередь для этого у крупных фирм и есть огромные патентные портфели и патентные пулы), но тролль к таким искам неуязвим, так как ничего не производит.

\subparagraph{Крупнейшие патентные тролли:}
По данным Business Insider (2012), крупнейшими по количеству патентов троллями являются: Intellectual Ventures (более 10 тысяч), Round Rock Research LLC (3,5 тыс.), Rockstar Consortium LLC (3,5 тыс.), Interdigital (3 тыс.), WARF (2,5 тыс.), Rambus (1,6 тыс.), Tessera Technologies Inc. (1,4 тыс.), Acacia Technologies (1,3 тыс.).

\subparagraph{Прецеденты}
\begin{itemize}
	\item В октябре 1999 года в России был опубликован патент за номером 2 139 818 на изобретение «Сосуда стеклянного», описание которого полностью соответствует обыкновенной стеклянной бутылке. Патентообладатель, ООО «Технополис», пытался потребовать от компаний, производящих пиво и безалкогольные напитки, лицензионных отчислений в размере не менее 0,5 \% от выручки. В настоящее время патент аннулирован по решению Палаты по патентным спорам.
	\item 22 января 2008 года американская компания Minerva Industries запатентовала устройство, по описанию напоминающее смартфон, и тут же подала иски против крупных производителей мобильных телефонов: Apple, Nokia, RIM, Sprint, AT&T, HP, Motorola, Helio, HTC, Sony Ericsson, UTStarcomm, Samsung и некоторых других. Что примечательно, первый раз иски были отправлены ещё до получения патента.
	\item В марте 2013 россиянин требует с Samsung \$10,9 млн за мобильные телефоны с двумя SIM-картами.
	\begin{nota}
		Оригинальный патент на 2-симочные коннекторы зарегистрирован 10 мая 2002 Quanta Computer, Inc. (Taoyuan, TW).
	\end{nota}
    \item В апреле 2019 года некий Андрей Дуксин зарегистрировал товарный знак сообщества SCP в России. Данное сообщество работает на лицензии Creative Commons BY-SA 3.0, соответственно прав на ТЗ Дуксин не имеет.
\end{itemize}

\newpage

\section{Основные группы потребителей на рынке инновационных продуктов, (согласно модели Джеффри А. Мура). Жизненный цикл принятия технологий на рынке инноваций.}\label{erste}

В тридцатые годы ХХ века Эверет Роджерс, изучавший вопросы коммуникации, разработал  теорию  диффузии  инноваций.  Он  предположил,  что  людей  по  их отношению  к  новым  идеям  и  по  очередности  принятия  нововведений  можно разделить на несколько категорий:
\begin{enumerate}
	\item \textbf{Инноваторы (2.5\%)} --- предприимчивые,  хорошо  образованные,  имеют  множество  источников информации, более склонны к риску.
	\item \textbf{Ранние последователи (13.5\%)} --- социальные лидеры, популярные, с хорошим образованием,  с  готовностью принимают/пробуют  нововведения,  но  более осторожны, чем инноваторы.
	\item \textbf{Раннее большинство (34\%)} --- рассудительные, более осторожные, чем ранние последователи,    но    принимающие    нововведение    раньше,    чем среднестатистический  последователь,  имеют  множество  неформальных социальных контактов.
	\item \textbf{Позднее  большинство (34\%)} --- скептики,  принимают  нововведение только, когда его уже приняло большинство.
	\item \textbf{Отстающие (16\%)} --- традиционалисты,  не  любят  перемены,  принимают  их только когда они стали общепринятой нормой, традицией; соседи и друзья  --- основные источники информации.
\end{enumerate}


Джеффри  А.  Мур  пошёл  дальше,  адаптировав  эту  теорию  к  рынку инновационных высокотехнологичных товаров. Дж. Мур выделяет два этапа в развитии рынка инноваций: \textit{ранний рынок} и \textit{основной рынок}.

\subparagraph{Жизненный цикл инновации}

\begin{enumerate}
	\item \textbf{Зарождение}

	--- Сопровождается выполнением необходимого объема научно-исследовательских и опытно-конструкторских работ, разработкой и созданием опытной партии новшества;

	--- Характеризуется медленным и растянутым во времени наращиванием выпуска продукции.
	\item \textbf{Рост} (промышленное освоение с одновременным выходом продукта на рынок)

	--- Характеризуется медленным и растянутым во времени наращиванием выпуска продукции.
	\item \textbf{Зрелость} (стадия серийного или массового производства и увеличение объема продаж)

	--- Отличается быстрым наращиванием производства, значительным увеличением загрузки производственных мощностей, отлаженностью технологического процесса и организации производства
	\item \textbf{Насыщение рынка} (максимальный объем производства и максимальный объем продаж)

	--- Характеризуется устойчивыми темпами наибольших объемов выпуска продукции и максимально возможной загрузкой производственных мощностей
	\item \textbf{Упадок} (свертывание производства и уход продукта с рынка)

	--- Связан с падением загрузки мощностей, сворачиванием производства данного товара и резким уменьшением товарных запасов вплоть до нуля
\end{enumerate}

\end{figure}

\newpage

\section{Особенности рынка инновационных продуктов.}\label{erste}

Инновационные продукты образуют специфический рынок наукоемкой и научно-технической продукции. Его особенности по сравнению с рынком «традиционных» товаров многообразны и затрагивают все стороны отношений, складывающиеся между продавцом и покупателем, требуя, соответственно, своего отражения в маркетинговой политике компании. Среди особенностей данного рынка следует выделить следующие:
\begin{itemize}
	\item[0^{\circ}:] особенности самого продукта (уникальность; порой технологическая сложность;
	\item[1^{\circ}:] высокие затраты на его производство на первых этапах;
	\item[2^{\circ}:] новизна рынка для фирмы (особенно для малой инновационной компании, находящейся на этапе start-up);
	\item[3^{\circ}:] неизвестность продукта (а, иногда, и фирмы-производителя) для рынка;
	\item[4^{\circ}:] непредсказуемость поведения потребителей;
	\item[5^{\circ}:] малая эластичность спроса от цены, поэтому ограниченное влияние ценовой политики на объемы продаж;
	\item[6^{\circ}:] небольшая емкость рынка (особенно для высокотехнологичной продукции производственного назначения);
	\item[7^{\circ}:] отсутствие прямых конкурентов (из-за монополии на интеллектуальную собственность);
	\item[8^{\circ}:] достижения сотрудников компании в теоретической области при грамотно построенной PR-политики могут существенно поднять рейтинг фирмы у потребителей;
	\item[9^{\circ}:] зависимость сбыта инновационной продукции от уровня инновационного потенциала потребителя: многие пионерные инновации трудно продаются из-за общей технологической отсталости ряда рынков сбыта.
\end{itemize}

Кроме того, рынок инноваций имеет свои особенные функции, такие как:
\begin{itemize}
	\item Финансирование прикладных научных исследований;
	\item Появление нового знания;
	\item Создание инновационных технологий, товаров и услуг;
	\item Удовлетворение потребностей предприятий реального сектора экономики в инновационных разработках;
	\item Распространение инноваций;
	\item Повышение конкурентоспособности организаций, региона, страны.
\end{itemize}
\newpage

\section{Источники финансирования инновационной деятельности на разных этапах жизненного цикла инновационной компании.}\label{erste}

\begin{wrapfigure}{r}{0.4\linewidth}
	\includegraphics[width=200pt, height=190pt, keepaspectratio]{jcurve.jpg}
	\caption{J-Curve (Жизненный цикл)}
\end{wrapfigure}
Существуют различные подходы к разделению жизненного цикла инновационных компаний на стадии. В соответствии с наиболее распространенным подходом выделяют следующие стадии: посевную, стартап, ранний рост, расширение и позднюю. На поздней стадии развития компании инвестор осуществляет выход, т. е. завершает свой инвестиционный цикл. В некоторых подходах отдельно от посевной выделяют предпосевную стадию.

\vspace{1em}
Для каждой стадии характерны определенные источники финансирования, выбор которых зависит от объема требуемых инвестиций, отрасли, в которой функционирует стартап, целей использования финансовых ресурсов, раунда финансирования и т.д. Потребность в инвестициях на посевной стадии и стадии «стартап» обычно ниже, чем наболее поздних, но ранние этапы являются решающими в становлении компании и определяют ее дальнейшее развитие.

\subparagraph{Посевная стадия} \\*

Как правило, финансирование данной стадии происходит за счет \textit{собственных средств основателей}, \textit{краудфандинга}, участия в конкурсах, также могут привлекаться \textit{государственные и частные гранты} и \textit{взносы частных инвесторов} (бизнес-ангелов).

\subparagraph{Стадия Стартапа} \\*

Крупные компании и фонды редко финансируют проекты на ранних стадиях жизненного цикла в связи со значительными рисками. Основными инвесторами проектов являются \textit{бизнес-ангелы} и \textit{посевные венчурные фонды}. При этом венчурные фонды заинтересованы в проектах с уже созданными рабочим прототипом/альфа-версией, которые можно протестировать на потребителях и собрать первичные отзывы. Данный подход позволяет снизить риски невостребованности продукта и его технической нереализуемости.

\subparagraph{Стадия раннего роста} \\*
На этом этапе основными источниками финансовых ресурсов
являются \textit{венчурные фонды} (государственные, частные, корпоративные), \textit{фонды прямых
инвестиций}, \textit{крупные корпорации} и другие институциональные инвесторы.

\subparagraph{Стадия расширения} \\*
Ключевыми источниками
финансирования, как и на предыдущей стадии, являются: \textit{венчурные фонды}
(государственные, частные, корпоративные), \textit{фонды прямых
инвестиций}, \textit{крупные корпорации}, \textit{портфельные} и другие \textit{институциональные инвесторы}. Также возможно
привлечение банковских продуктов и портфельных инвесторов (например, \textit{pre-IPO}
фонды).

\subparagraph{Поздняя стадия} \\*
Инвестор осуществляет
выход из ее капитала. В зависимости от специфики компании и рынка определяется
оптимальный механизм выхода, например, \textit{продажа стратегическому или финансовому
инвестору}, \textit{выкуп менеджментом компании}, \textit{публичное размещение акций}, \textit{ликвидация}.
\newpage

\section{Основные подходы к оценке инновационной компании.}\label{erste}

На практике количественная оценка инновационной активности предприятий осуществляется на основе трех основных подходов: функционального, результатного и факторно-результатного

\subparagraph{Функциональный подход} отличается той основной особенностью, что в его рамках для оценки инновационной активности предприятия используются не характеристики конечных результатов его инновационной деятельности, а показатели интенсивности осуществления предприятием тех или иных видов или компонентов такой деятельности, в частности:
\begin{itemize}
	\item реализация предприятием отдельных видов и стадий НИОКР;
	\item приобретение овеществленных новых технологий (различных видов нового технологического оборудования и оснастки);
	\item приобретение неовеществленных новых технологий (различных видов объектов интеллектуальной собственности);
	\item обучение и переподготовка персонала;
	\item осуществление элементов комплекса маркетинга для новых видов продукции.
\end{itemize}
Оценка каждого перечисленных видов деятельности в рамках функционального подхода осуществляется с помощью натуральных и стоимостных (затратных) показателей.

\subparagraph{Результатный подход} основывается на получении оценки инновационной активности предприятия с помощью показателей, характеризующих различные аспекты конечных результатов его инновационной деятельности. В зависимости от характера таких аспектов, данный подход подразделяется на три более частных подхода: динамический, эффективностный и смешанный динамико-эффективностный.
\begin{itemize}
	\item[(a)] \textit{Динамический подход}
	\item[(b)] \textit{Эффективностный подход}
	\item[(c)] \textit{Смешанный динамико-эффективностный подход}
\end{itemize}

\subparagraph{Факторно-результатный подход}
к оценке инновационной активности предприятия отличается тем, что его рамках подобная оценка осуществляется на основе совмещения факторных и результатных характеристик инновационной деятельности.

Для оценки факторных параметров инновационной активности в рамках данного подхода обычно используется комплекс таких критериев, как: объем затрат предприятия на осуществление НИОКР, приобретение ОИС и финансирование межфирменных исследовательских проектов; показатели состава и числа сотрудников, временных групп, подразделений и межфирменных объединений, занятых в осуществлении НИОКР; объем новых технологий, приобретенных предприятием в рамках систем технологического трансфера; масштабы и качественный уровень материальной базы научно-исследовательской деятельности предприятия и др.

В качестве критериев оценки результатных характеристик инновационной активности в рамках данного подхода используются как показатели обычного результатного подхода, так и ряд специфических показателей, в частности: показатели длительности отдельных стадий инновационных разработок; показатели динамики обновления портфеля продукции предприятия; объем новых технологий, переданных предприятием в рамках систем технологического трансфера; объемы экспортируемой предприятием новой продукции; число внедренных за период новых технологий и видов продукции и т.д.

\newpage

\section{Общие и различные черты между бизнес-ангельским и венчурным финансированием инновационной деятельности.}\label{erste}

\subsection{Общие черты}

\textbf{Бизнес-ангельское} и \textbf{венчурное финансирование} — смежные понятия: в бизнес-практике используют общую терминологию, способы и методы работы. Принцип такого финансирования — вложение капитала в компанию, у которой есть потенциал роста, в обмен на значительную (выше 10-15\%) долю с целью получения высокой прибыли посредством продажи этой доли через определенное время.


Обычно инвесторы вкладываются в инновационную компанию с интересной бизнес-идеей, нематериальными активами, обладающую интеллектуальной собственностью.

\subsection{Различия}

В сотрудничестве с бизнес-ангелами существенную роль играет психологическая совместимость собственников компании и инвестора. Все это очень близко венчурному инвестированию, но есть отличия:

\vspace{1em}
\begin{tabulary}{0.94\linewidth}{|J|J|J|}
	\hline
	"" & \textbf{Бизнес-ангел} & \textbf{Венчурный инвестор} \\ \hline
	\textbf{Формирование компании как юридического лица:} & в стадии формирования & работы по формированию завершены \\ \hline
	\textbf{Команда:} & формируется, известна лишь часть предполагаемых членов & сформирована \\ \hline
	\textbf{Рынок:} & изучается, есть общее видение, отдельные наработки. & выполнены маркетинговые исследования, потребители определены \\ \hline
	\textbf{Интеллектуальная собственность:} & разработки не завершены, отсутствует регистрация. & первоначальный пакет сформирован и защищен \\ \hline
	\textbf{Производство:} & налаживается опытное производство. & налажено опытное производство, ведутся работы по налаживанию массового \\ \hline
	\textbf{Будущий денежный поток:} & не поддается прогнозированию, есть только оценки. & прогнозируется с большой степенью достоверности \\ \hline
	\textbf{Бизнес-план как документ:} & имеется общее видение бизнеса и отдельные элементы плана. & есть четкий бизнес-план, который содержит экономическое обоснование проекта \\ \hline
	\textbf{Характер отношений:} & частые неформальные встречи. & встречи в рамках установленных процедур в оговоренные сроки \\ \hline
\end{tabulary}

\begin{proposition}[Венчурное инвестирование]
	Предоставление средств молодым компаниям, находящимся на ранних стадиях развития, на долгий срок в обмен на долю в этих компаниях.
\end{proposition}

Венчурный инвестор обычно не стремится приобрести контрольный пакет акций компании, во всяком случае, при первичном инвестировании. В этом его основное отличие от стратегического инвестора или партнера, который хочет установить контроль над компанией.

Цель венчурного капиталиста иная: приобретая пакет акций или долю, меньшую, чем контрольный пакет (Minority Position), инвестор рассчитывает, что менеджмент компании будет использовать его деньги в качестве финансового рычага (Financial Leverage) для быстрого роста и развития своего бизнеса.
\newpage

\section{Основные источники конфликтов между венчурным инвестором и предпринимателем в процессе структурирования инвестиционной сделки.}\label{erste}

\newpage

\section{Основные риски инновационного проекта, включая риски выделяемые Дж. Эндрю (концепция 4S факторов).}\label{erste}

Принятие   любого   инновационного   решения   сопряжено   с   некой неопределенностью  достижения  результата,  т.е.  с  риском  (\textbf{риск} --- это  разница между  ожидаемым  и  реальным  результатом).  На  принятие  инновационных решений  влияет  значительное  количество  рисков,  как  систематических,  так  и несистематических.

\subparagraph{Инновационные риски:}
\begin{itemize}
	\item Риски ошибочного выбора инновационного проекта;
	\item Риски  необеспечения  инновационного  проекта  достаточным  уровнем финансирования;
	\item Маркетинговые риски текущего снабжения ресурсами, необходимыми для реализации инновационного проекта;
	\item Маркетинговые риски сбыта результатов инновационного проекта;
	\item Риски неисполнения хозяйственных договоров;
	\item Риски возникновения непредвиденных затрат и снижения доходов;
	\item Риски усиления конкуренции;
	\item Риски,   связанные   с   обеспечением   прав   собственности   на инновационный проект.
\end{itemize}

Т.к. учет факторов рисков в инвестиционном проектировании сводится к оценке уровня рисков и разработке мероприятий по их снижению, то при  их  учете  пригодны  классические  методы  и  приемы  управления  рисками. Однако,  из-за  специфичности  самих  инновационных  решений  им  свойственны особые   несистематические   инновационные   риски.

\newpage

\section{Основные подходы по оценке эффективности на разных этапах развития инновационной компании.}\label{erste}

\newpage

\section{Нефинансовые показатели, влияющие на инвестиционную привлекательность инновационной компании.}\label{erste}

Условно факторы инвестиционной привлекательности инновационных проектов и программ можно разделить на две группы: финансово-экономические и внеэкономические. В большинстве случаев инвестора привлекают высокие финансовые показатели, однако существуют ситуации, при которых инноватор вынужден реализовать новшество несмотря на его прямую экономическую непривлекательность.

\subparagraph{Нефинансовые показатели:}
\begin{itemize}
	\item наличие  факторов успешности нового товара (превосходство товара над конкурирующими)
	\item маркетинговые   преимущества   компании (размер   потенциального   рынка, понимание клиентов)
	\item технологические преимущества самого проекта
	\item риск непредсказуемости реакции покупателей
\end{itemize}

\newpage

\section{До- и постинвестиционная стоимость компании.}\label{erste}

\begin{proposition}[Доинвестиционная стоимость компании]
	Cтоимость компании, согласованная между существующими   владельцами   и   новыми   инвесторами,   и определяемая непосредственно перед тем, как в нее будет сделана венчурная инвестиция.
\end{proposition}

Предварительная  стоимость  компании  показывает  целесообразность вложений  в  акции  оцениваемой  компании,  а  значит,  рассчитывается  с  точки зрения  привлекательности  для  инвестора.  Понятно,  что  компании  выгодно создавать положительный образ у инвестора, и для этого менеджеры стараются завысить стоимость. Предварительная оценка стоимости компании зависит от:
\begin{itemize}
	\item общей  характеристики  отрасли  и  компании  (ее  размера,  выпускаемой продукции), ее партнеров, конкурентов и клиентов;
	\item состояния  финансового  пакета  документов  (устава,  учредительных документов,  бухгалтерского  баланса,  отчета  о  прибылях  и  убытках)
\end{itemize}

Расчет первоначальной стоимости необходим для определения структуры сделки, то есть объема привлекаемых средств, количества и цены выпускаемых акций. Учитывая, что объем привлекаемых средств должен иметь обоснование и соответствовать нуждам  компании,  предполагается,  что  объем  инвестиций должен быть пропорционален первоначальной оценке стоимости компании.

\begin{proposition}[Послеинвестиционная стоимость компании]
	оценка  стоимости компании,  произведенная  сразу  же  после  очередной  стадии  финансирования  с учетом вложенных средств.
\end{proposition}

Если  компания  публична,  то  ее  постинвестиционная  стоимость  является рыночной стоимостью или стоимостью бизнеса компании. Постинвестиционная стоимость компании имеет важное значение для инвесторов, так как показывает то,  как  рынок  оценивает  бизнес  в  целом. Таким  образом,  предварительная  и постинвестиционная стоимостькомпании важна для определения:
\begin{itemize}
	\item объема средств, привлекаемых путем венчурного финансирования
	\item при выходе из инвестиций
	\item определения цены каждой акции
	\item доли собственности инвестора
\end{itemize}
\newpage

\section{Понятие cap table (capitalization table) для стартапов.}\label{erste}

\begin{proposition}[Cap table]
	Таблица, как правило, для начинающего или раннего этапа венчурного бизнеса, которое показывает капитализацию или доли владения в компании, в том числе акции, привилегированные акции и опционы, а также различные цены, уплаченные заинтересованными сторонами за эти ценные бумаги.
\end{proposition}
В таблице используются эти данные, чтобы показать доли владения на полностью разведенной основе, тем самым позволяя определить общую структуру капитала компании с одного взгляда. Учредители обычно перечисляются первыми, за ними следуют руководители и ключевые сотрудники с долями акций, затем инвесторы, такие как ангельские инвесторы и фирмы венчурного капитала, и другие, которые участвуют в бизнес-плане.

\section{Виды и особенности презентаций бизнес-проектов.}\label{erste}

Под бизнес-презентацией понимают переговоры представителей разных фирм по рассмотрению бизнес-вопросов. В зависимости от рассматриваемых вопросов презентации могут иметь следующие цели:
\begin{itemize}
	\item \textbf{Презентация фирмы}

	Целями такой презентации являются: создание имиджа фирмы среди деловых кругов, создание или воссоздание благоприятного образа фирмы, реклама имени фирмы. По сути своей такая презентация является частью рекламной кампании организации.
	\item \textbf{Презентация бизнес-проекта}

	Цель этого вида презентации - информирование людей о каком-либо проекте, определение обратной реакции к проекту, поиск заинтересованных в поддержке разработки и реализации проекта. Этот вид презентации наиболее требователен к форме подачи, содержанию и подготовке), т. к. предполагает убеждение аудитории в необходимости осуществления разработки или воплощения проекта.
	\item \textbf{Отчет о выполненных работах}

	Цель - ознакомить, предоставить определенной узкой группе людей результаты работ. Такая презентация менее требовательна к выполнению определенных правил подготовки и вполне может быль спонтанной, если необходимые данные у вас под рукой и содержатся в полном порядке.
	\item \textbf{Обсуждение плана будущих работ}

	Такая презентация аналогична предыдущему виду презентаций, только объект здесь будущие работы организации или личности. Целями её могут являться: информирование определенного круга лиц о намеченных работах, описание намеченных работ с целью подтверждения объекта презентации критическому анализу и изменению.
	\item \textbf{Презентация товара}

	Цели такой презентации ясны: создание знания о новой марке, товаре или услуге на целевом рынке, ознакомление потребителей с новыми возможностями товара, расписания магазина и т.д., достижение предпочтения марке и т.п. Такие презентации могут быть внутренние и внешние презентации.

\end{itemize}
\end{document}
