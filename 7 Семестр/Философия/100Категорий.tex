\documentclass[a4paper, 12pt]{article}

%%% Матпакет
\usepackage{amsmath, amssymb, amscd, amsthm, amsfonts}
\usepackage{hyperref}
\usepackage{icomma}                  % "Умная" запятая: $0,2$ --- число, $0, 2$ --- перечисление

%%% Страница
\usepackage{extsizes} % Возможность сделать 14-й шрифт
\usepackage{geometry} % Простой способ задавать поля
	\geometry{top=25mm}
	\geometry{bottom=25mm}
	\geometry{left=18mm}
	\geometry{right=14mm}
\usepackage{indentfirst}

%%%Стили
\usepackage{xcolor}
%\usepackage{sectsty}
%\allsectionsfont{\sffamily}
\usepackage{titlesec, blindtext, color} % подключаем нужные пакеты

%%% Работа с русским языком
\usepackage{cmap}					% поиск в PDF
\usepackage{mathtext} 				% русские буквы в фомулах
\usepackage[T2A]{fontenc}			% кодировка
\usepackage[utf8]{inputenc}			% кодировка исходного текста
\usepackage[english, russian]{babel}	% локализация и переносы

\usepackage{soul} % Модификаторы начертания
\usepackage{csquotes} % Цитаты

%%% Теоремы
\theoremstyle{plain} % Это стиль по умолчанию, его можно не переопределять.
\newtheorem{theorem}{Теорема}[section]
\newtheorem{proposition}[theorem]{Утверждение}
\newtheorem{corollary}{Следствие}

\theoremstyle{definition} % "Утверждение"
\newtheorem{definition}{Определение}
\newtheorem{problem}{Задача}[section]

\theoremstyle{remark} % "Примечание"
\newtheorem{example}{Пример}
\newtheorem{nota}{Примечание}

%%% Работа с картинками
\usepackage{graphicx}                % Для вставки рисунков
\graphicspath{{img/}}  				% папки с картинками
\setlength\fboxsep{3pt}              % Отступ рамки \fbox{} от рисунка
\setlength\fboxrule{1pt}             % Толщина линий рамки \fbox{}
\usepackage{wrapfig}                 % Обтекание рисунков текстом
\title{Философия \\ Категориальный минимум}
\author{ЭФ МГУ}
\date{зимняя сессия 2020}

%%% Работа с таблицами
\usepackage{array,tabularx,tabulary,booktabs} % Дополнительная работа с таблицами
\usepackage{longtable}                        % Длинные таблицы
\usepackage{multirow}                         % Слияние строк в таблице

\usepackage{hyperref}
\usepackage[usenames, dvipsnames, svgnames, table, rgb]{color}
\hypersetup{				                % Гиперссылки
    unicode=true,                           % русские буквы в раздела PDF
    pdftitle={Заголовок},                   % Заголовок
    pdfauthor={Автор},                      % Автор
    pdfsubject={Тема},                      % Тема
    pdfcreator={Создатель},                 % Создатель
    pdfproducer={Производитель},            % Производитель
    pdfkeywords={keyword1} {key2} {key3},   % Ключевые слова
    colorlinks=true,        	            % false: ссылки в рамках; true: цветные ссылки
    linkcolor=red,                          % внутренние ссылки
    citecolor=green,                        % на библиографию
    filecolor=magenta,                      % на файлы
    urlcolor=cyan                           % на URL
}

\begin{document}

\maketitle

\subparagraph{Абсолют}
    первоначало всего сущего, которое не зависит ни от чего другого, само содержит все существующее и творит его ленного соединения его отдельных частей

\subparagraph{Абстракция}
    одна из сторон, форм познания, заключающаяся в мысленном отвлечении от ряда свойств предметов и отношений между ними и выделении, вычленении свойства или отношения

\subparagraph{Агностицизм}
    учение, отрицающее полностью или частично возможность познания мира

\subparagraph{Аксиология}
    философское учение о природе ценностей, их месте в реальности и о структуре ценностного мира, т. е. о связи различных ценностей между собой, с социальными и культурными факторами и структурой личности

\subparagraph{Акциденция}
    временное, преходящее, несущественное свойство вещи, в отличие от существенного, субстанциального

\subparagraph{Методы научного познания}
    \begin{itemize}
        \item[(a)] \emph{Анализ} --- это мысленное или физическое расчленение исследуемого объекта на составные части и изучение их в отдельности
         \item[(б)] \emph{Синтез} --- это формирование целостного образа объекта на основе мысленного соединения его отдельных частей
    \end{itemize}

\subparagraph{Архетип}
    изначальные, врожденные образы (мотивы), составляющие содержание так называемого коллективного бессознательного и лежащие в основе общечеловеческой символики сновидений, мифов, сказок и прочих созданий фантазии, в том числе и художественной

\subparagraph{Атрибут}
    неотъемлемое свойство предмета, без которого предмет не может ни существовать, ни мыслиться

\subparagraph{Бессознательное}
    совокупность психических состояний и процессов, которые осуществляются без участия сознания

\subparagraph{Бытие}
    существующий независимо от сознания объективный мир

\subparagraph{Вера}
    принятие чего-либо за истину, не нуждающееся в необходимом полном подтверждении истинности принятого со стороны чувств и разума и, следовательно, не могущее претендовать на объективную значимость

\subparagraph{Вероятность}
    количественная мера возможности осуществления события при наличии неопределенности, т.е. в ситуации, когда это событие характеризуется как возможное

\subparagraph{Философские категории, логически описывающие движение, способ существования материи во времени}
    \begin{itemize}
        \item[(a)] \emph{Возможность} --- это то, что может возникнуть и существовать при определённых условиях, стать действительностью
        \item[(b)] \emph{Действительность} --- это то, что уже возникло, существует
    \end{itemize}

\subparagraph{Время}
    форма протекания всех механических, органических и психических процессов, условие возможности движения, изменения, развития

\subparagraph{Гедонизм}
    этическая установка, с точки зрения которой основой природы человека является его стремление к наслаждению, а потому все ценности и ориентации деятельности должны быть подчинены или сведены к наслаждению как подлинному высшему благу.

\subparagraph{Гилозоизм}
    философская концепция, признающая одушевленность всех тел, космоса, материи, природы

\subparagraph{Гипотеза}
    система умозаключений, посредством которой на основе ряда фактов делается вывод о существовании объекта, связи или причины явления, причем вывод этот нельзя считать абсолютно достоверным

\subparagraph{Гносеология}
    учение о познании; раздел философии, рассматривающий проблемы человеческого познания, вопросы о его возможностях и границах, о путях и средствах достижения истинного знания, о роли познания в бытии человека


\end{document}
