\documentclass[a4paper, 12pt]{article}

%%% Матпакет
\usepackage{amsmath, amssymb, amscd, amsthm, amsfonts}
\usepackage{hyperref}
\usepackage{icomma}                  % "Умная" запятая: $0,2$ --- число, $0, 2$ --- перечисление

%%% Страница
\usepackage{extsizes} % Возможность сделать 14-й шрифт
\usepackage{geometry} % Простой способ задавать поля
	\geometry{top=25mm}
	\geometry{bottom=25mm}
	\geometry{left=18mm}
	\geometry{right=14mm}
\usepackage{indentfirst}

%%%Стили
\usepackage{xcolor}
%\usepackage{sectsty}
%\allsectionsfont{\sffamily}
\usepackage{titlesec, blindtext, color} % подключаем нужные пакеты

%%% Работа с русским языком
\usepackage{cmap}					% поиск в PDF
\usepackage{mathtext} 				% русские буквы в фомулах
\usepackage[T2A]{fontenc}			% кодировка
\usepackage[utf8]{inputenc}			% кодировка исходного текста
\usepackage[english, russian]{babel}	% локализация и переносы

\usepackage{soul} % Модификаторы начертания
\usepackage{csquotes} % Цитаты

%%% Теоремы
\theoremstyle{plain} % Это стиль по умолчанию, его можно не переопределять.
\newtheorem{theorem}{Теорема}[section]
\newtheorem{proposition}[theorem]{Утверждение}
\newtheorem{corollary}{Следствие}

\theoremstyle{definition} % "Утверждение"
\newtheorem{definition}{Определение}
\newtheorem{problem}{Задача}[section]

\theoremstyle{remark} % "Примечание"
\newtheorem{example}{Пример}
\newtheorem{nota}{Примечание}

%%% Работа с картинками
\usepackage{graphicx}                % Для вставки рисунков
\graphicspath{{img/}}  				% папки с картинками
\setlength\fboxsep{3pt}              % Отступ рамки \fbox{} от рисунка
\setlength\fboxrule{1pt}             % Толщина линий рамки \fbox{}
\usepackage{wrapfig}                 % Обтекание рисунков текстом
\title{Философия \\ Категориальный минимум}
\author{ЭФ МГУ}
\date{зимняя сессия 2020}

%%% Работа с таблицами
\usepackage{array,tabularx,tabulary,booktabs} % Дополнительная работа с таблицами
\usepackage{longtable}                        % Длинные таблицы
\usepackage{multirow}                         % Слияние строк в таблице

\usepackage{hyperref}
\usepackage[usenames, dvipsnames, svgnames, table, rgb]{color}
\hypersetup{				                % Гиперссылки
    unicode=true,                           % русские буквы в раздела PDF
    pdftitle={Заголовок},                   % Заголовок
    pdfauthor={Автор},                      % Автор
    pdfsubject={Тема},                      % Тема
    pdfcreator={Создатель},                 % Создатель
    pdfproducer={Производитель},            % Производитель
    pdfkeywords={keyword1} {key2} {key3},   % Ключевые слова
    colorlinks=true,        	            % false: ссылки в рамках; true: цветные ссылки
    linkcolor=red,                          % внутренние ссылки
    citecolor=green,                        % на библиографию
    filecolor=magenta,                      % на файлы
    urlcolor=cyan                           % на URL
}

\begin{document}

\maketitle

\subparagraph{Абсолют}
    первоначало всего сущего, которое не зависит ни от чего другого, само содержит все существующее и творит его ленного соединения его отдельных частей

\subparagraph{Абстракция}
    одна из сторон, форм познания, заключающаяся в мысленном отвлечении от ряда свойств предметов и отношений между ними и выделении, вычленении свойства или отношения

\subparagraph{Агностицизм}
    учение, отрицающее полностью или частично возможность познания мира

\subparagraph{Аксиология}
    философское учение о природе ценностей, их месте в реальности и о структуре ценностного мира, т. е. о связи различных ценностей между собой, с социальными и культурными факторами и структурой личности

\subparagraph{Акциденция}
    временное, преходящее, несущественное свойство вещи, в отличие от существенного, субстанциального

\subparagraph{Методы научного познания}
    \begin{itemize}
        \item[(a)] \emph{Анализ} --- это мысленное или физическое расчленение исследуемого объекта на составные части и изучение их в отдельности
         \item[(б)] \emph{Синтез} --- это формирование целостного образа объекта на основе мысленного соединения его отдельных частей
    \end{itemize}

\subparagraph{Архетип}
    изначальные, врожденные образы (мотивы), составляющие содержание так называемого коллективного бессознательного и лежащие в основе общечеловеческой символики сновидений, мифов, сказок и прочих созданий фантазии, в том числе и художественной

\subparagraph{Атрибут}
    неотъемлемое свойство предмета, без которого предмет не может ни существовать, ни мыслиться

\subparagraph{Бессознательное}
    совокупность психических состояний и процессов, которые осуществляются без участия сознания

\subparagraph{Бытие}
    существующий независимо от сознания объективный мир

\subparagraph{Вера}
    принятие чего-либо за истину, не нуждающееся в необходимом полном подтверждении истинности принятого со стороны чувств и разума и, следовательно, не могущее претендовать на объективную значимость

\subparagraph{Вероятность}
    количественная мера возможности осуществления события при наличии неопределенности, т.е. в ситуации, когда это событие характеризуется как возможное

\subparagraph{Философские категории, логически описывающие движение, способ существования материи во времени}
\begin{itemize}
	\item[(a)] \emph{Возможность} --- это то, что может возникнуть и существовать при определённых условиях, стать действительностью
	\item[(b)] \emph{Действительность} --- это то, что уже возникло, существует
\end{itemize}

\subparagraph{Время}
    форма протекания всех механических, органических и психических процессов, условие возможности движения, изменения, развития

\subparagraph{Гедонизм}
    этическая установка, с точки зрения которой основой природы человека является его стремление к наслаждению, а потому все ценности и ориентации деятельности должны быть подчинены или сведены к наслаждению как подлинному высшему благу.

\subparagraph{Гилозоизм}
    философская концепция, признающая одушевленность всех тел, космоса, материи, природы

\subparagraph{Гипотеза}
    система умозаключений, посредством которой на основе ряда фактов делается вывод о существовании объекта, связи или причины явления, причем вывод этот нельзя считать абсолютно достоверным

\subparagraph{Гносеология}
    учение о познании; раздел философии, рассматривающий проблемы человеческого познания, вопросы о его возможностях и границах, о путях и средствах достижения истинного знания, о роли познания в бытии человека

\subparagraph{Движение}
	понятие процессуального феномена, охватывающего все типы изменений и взаимодействий

\subparagraph{Общенаучные логические методы получения нового знания}
\begin{itemize}
	\item[(a)] \emph{Дедукция} --- вид умозаключения, состоящий в движении мысли от общего к частному.
	\item[(b)] \emph{Индукция} --- вид умозаключения, состоящий в движении мысли от частного к общему.
\end{itemize}

\subparagraph{Деизм}
	учение, которое признает существование бога в качестве безличной первопричины мира, развивающегося затем по своим собственным законам

\subparagraph{Детерминизм}
	философское учение об объективной закономерной взаимосвязи и взаимообусловленности явлений материального и духовного мира

\subparagraph{Деятельность}
	один из важнейших атрибутов бытия человека, связанный с целенаправленным изменением внешнего мира, самого человека

\subparagraph{Диалектика}
	наука о всеобщих законах развития природы, общества, человека и мышления

\subparagraph{Догма}
	 положение, принимающее слепо на веру, без доказательств

\subparagraph{Дуализм}
	философское учение, считающее, в противоположность монизму, материальную и духовную субстанции равноправными началами.

\subparagraph{Дух}
	философское понятие, означающее невещественное начало, в отличие от материального, природного начала.

\subparagraph{Душа}
	Этим понятием в истории философии выражалось воззрение на внутренний мир человека, отождествляемый в идеализме с особой нематериальной субстанцией.

\subparagraph{Жизнь}
	специфическая форма организации материи, характеризующаяся единством трех моментов:
	\begin{itemize}
		\item[1^\circ:] наследственной программой, записанной в совокупности генов \emph{(геном)}
		\item[2^\circ:] обменом веществ, специфика которого определяется наследственной программой;
		\item[3^\circ:] самовоспроизведением в соответствии с этой программой
	\end{itemize}

\subparagraph{Закон}
	внутренняя существенная и устойчивая связь явлений, обусловливающая их упорядоченное изменение

\subparagraph{Знак}
	материальный, чувственно воспринимаемый предмет (явление, действие), который выступает как представитель др. предмета, свойства или отношения

\subparagraph{Знание}
	результат процесса познания, постижения действительного мира человеком, адекватное отражение объективной действительности в сознании в виде создаваемых в процессе познания представлений, понятий, суждений, теорий

\subparagraph{Идеальное}
	субъективный образ объективной реальности, возникающий в процессе целесообразной деятельности человека

\subparagraph{Идеология}
	система концептуально оформленных представлений и идей, которая выражает интересы, мировоззрение и идеалы различных субъектов политики и выступает формой санкционирования или существующего в обществе господства и власти, или радикального их преобразования

\subparagraph{Имманентное}
	внутренне присущее тому или иному предмету, явлению или процессу свойство (закономерность)

\subparagraph{Инстинкт}
	естественное влечение; свойственная роду и виду врожденная, т.е. наследственная, склонность к определенному поведению, или образу действий. Осуществляется автоматически или вследствие внешнего раздражения.

\subparagraph{Интерпретация}
	 истолкование, разъяснение смысла какой-либо знаковой системы.

\subparagraph{Интроспекция}
	наблюдение за собственными внутренними психическими явлениями, самонаблюдение

\subparagraph{Истина}
	знание, соответствующее действительности, процесс верного, адекватного отражения действительности в сознании человека, соответствие полученных знаний объекту познания

\subparagraph{Философско-методологические категории, характеризующие отношение между исторически развивающейся объективной действительностью и ее воспроизведением средствами научно-теоретического познания}
\begin{itemize}
	\item[(a)] \emph{Историческое} --- раскрывает конкретные особенности развития данного объекта, показывает его хронологию, выявляет его уникальные индивидуальные особенности
	\item[(b)] \emph{Логическое} --- отражает вещи и явления в обобщенном виде, подчеркивает нормативные и объективные стороны рассматриваемого объекта, дает его теоретическое понятие, выявляет его сущность в системе абстракций
\end{itemize}

\subparagraph{Качество}
	философская категория, выражающая существенную определенность предмета, благодаря которой он существует именно как такой, а не иной предмет

\subparagraph{Количество}
	важнейшая категория философии и науки, обозначающая численную или структурную определенность предметов, процессов и явлений, относительно безразличную их качественной определенности

\subparagraph{Логос}
	философский термин, фиксирующий единство понятия, слова и смысла, причем слово понимается в данном случае не столько в фонетическом, сколько в семантическом плане, а понятие - как выраженное вербально

\subparagraph{Материя}
	 фундаментальная исходная категория философии, обозначает объективную реальность, единственную субстанцию со всеми ее свойствами, законами строения и функционирования, движения и развития

\subparagraph{Метафизика}
	 философское учение о первичных основах всякого бытия или о сущности мира

\subparagraph{Методология}
	учение о способах организации и построения теоретической и практической деятельности человека

\subparagraph{Мировоззрение}
	система человеческих знаний о мире и о месте человека в мире, выраженная в аксиологических установках личности и социальной группы, в убеждениях относительно сущности природного и социального мира

\subparagraph{Мистика}
	То, что находится за пределами человеческого понимания, но несет в себе особый скрытый смысл

\subparagraph{Монизм}
	философское учение, к-рое принимает за основу всего существующего одно начало. Материалисты началом, основой мира считают материю

\subparagraph{Мышление}
	активный процесс отражения объективного мира в понятиях, суждениях, теориях, связанный с решением тех или иных задач, с обобщением и способами опосредствованного познания действительности

\subparagraph{Наблюдение}
	 целенаправленное и организованное восприятие внешнего мира, доставляющее первичный материал для научного исследования

\subparagraph{Соотносительные философские понятия}
\begin{itemize}
	\item[(a)] \emph{Необходимость} --- закономерный тип связи явлений, определяемый их устойчивой внутренней основой и совокупностью существенных условий их возникновения и развития
	\item[(b)] \emph{Случайность} --- такой тип связи, который обусловлен несущественными, внешними для данного явления, причинами абстракций
\end{itemize}

\subparagraph{Нигилизм}
	мировоззренческая система и социально-философская концепция абсолютного отрицания ценностей культуры, религиозных и нравственных норм, общественных институтов, исторического прошлого

\subparagraph{Общество}
	обособившаяся от природы но тесно связанная с ней часть материального мира, которая состоит из индивидуумов, наделенных волей и сознанием, включает в себя способы взаимодействия людей и формы их объединения.

\subparagraph{Онтология}
	учение о бытии, о принципах его строения, законах и формах.

\subparagraph{Опредмечивание}
	это процесс, в котором человеческие способности переходят в предмет и воплощаются в нем, благодаря чему предмет становится социально-культурным, или «человеческим предметом»

\subparagraph{Пантеизм}
	философское учение, согласно которому Бог и природа рассматриваются как близкие или тождественные понятия

\subparagraph{Парадигма}
	 совокупность устойчивых и общезначимых норм, теорий, методов, схем научной деятельности, предполагающая единство в толковании теории, в организации эмпирических исследований и интерпретации научных исследований

\subparagraph{Понятие}
	форма мысли, обобщенно отражающая предметы и явления посредством фиксации их существенных свойств

\subparagraph{Практика}
	материальная, целеполагающая деятельность человека, имеющая своим содержанием освоение и преобразование объективной действительности; всеобщая основа развития человеческого общества и познания

\subparagraph{Пространство}
	 гомогенная и бесконечная среда, в которой расположены воспринимаемые нами объекты

\subparagraph{Психика}
	системное свойство высокоорганизованной материи
	(на уровне животных и человека), заключающееся в активном отражении особью или индивидуумом объективного мира

\subparagraph{Разум}
	это высшая ступень познавательной деятельности человека, способность логически и творчески мыслить

\subparagraph{Редукция}
	методологический прием, широко используемый в научном познании как средство понимания и объяснения.

\subparagraph{Релятивизм}
	философская концепция, утверждающая относительность, условность и субъективность человеческого познания

\subparagraph{Рефлексия}
	 способность человеческого мышления к критическому самоанализу

\subparagraph{Самосознание}
	выделение человеком себя из объективного мира, осознание и оценка своего отношения к миру, себя как личности, своих поступков, действий, мыслей и чувств, желаний и интересов

\subparagraph{Свобода}
	 способность человека принимать решения и совершать поступки в соответствии со своими целями, интересами, идеалами

\subparagraph{Сенсуализм}
	 учение в гносеологии, признающее ощущение единственным источником познания

\subparagraph{Синкретизм}
	сочетание разнородных воззрений, взглядов, при котором игнорируется необходимость их внутреннего единства и непротиворечия друг другу

\subparagraph{Система}
	совокупность элементов, находящихся в отношениях и связях между собой и образующих определенную целостность, единство

\subparagraph{Сознание}
	совокупность психических процессов, высшая функция мозга, активно участвующая в осмыслении человеком объективного мира, а также собственного бытия.

\subparagraph{Солипсизм}
	субъективно-идеалистическая теория, согласно к-рой существует только человек и его сознание, а объективный мир, в т. ч. и люди, существует лишь в сознании индивида

\subparagraph{Структура}
	внутреннее устройство объекта, совокупность устойчивых внутренних связей, обеспечивающих целостность объекта и его тождественность самому себе

\subparagraph{Субстанция}
	философское понятие классической традиции для обозначения объективной реальности в аспекте внутреннего единства всех форм ее саморазвития

\subparagraph{Фундаментальные категории философии}
\begin{itemize}
	\item[(a)] \emph{Субъект} --- носитель деятельности, сознания и познания
	\item[(b)] \emph{Объект} --- то на что направлена реальная и познавательная активность субъекта
\end{itemize}

\subparagraph{Сущее}
	все, что так или иначе существует, обладает бытием независимо от категориального статуса или формы существования

\subparagraph{Существование}
	все многообразие изменчивых вещей в их связи и взаимодействии

\subparagraph{Философские категории, отражающие всеобщие необходимые стороны всех объектов и процессов в мире}
\begin{itemize}
	\item[(a)] \emph{Сущность} --- внутреннее содержание предмета, выражающееся в устойчивом единстве всех многообразных и противоречивых форм его бытия
	\item[(b)] \emph{Явление} --- то или иное обнаружение предмета, внешние формы его существования
\end{itemize}

\subparagraph{Творчество}
	процесс человеческой деятельности, создающий качественно новые материальные и духовные ценности

\subparagraph{Теодицея}
	религиозно-философское учение, цель которой сводится к оправданию представлений о Боге как абсолютном добре, сняв с него ответственность за наличие зла в мире

\subparagraph{Теория}
	совокупность высказываний, замкнутых относительно логического следования


\subparagraph{Традиция}
	способ бытия и воспроизводства элементов социального и культурного наследия, фиксирующий устойчивость и преемственность опыта поколений, времен и эпох

\subparagraph{Трансцендентный}
	понятие, введенное средневековыми схоластами для обозначения того, что находится за пределами чувственного опыта и рассудка человека (прежде всего религиозная реальность)

\subparagraph{Утилитаризм}
	этическая теория, признающая полезность поступка полным критерием его нравственности

\subparagraph{Утопия}
	понятие для обозначения описаний воображаемого идеального общественного строя, а также сочинений, содержащих нереальные планы социальных преобразований

\subparagraph{Факт}
\begin{itemize}
	\item[(a)]  синоним понятий истина, событие, результат; нечто реальное в противоположность вымышленному; конкретное, единичное в отличие от абстрактного и общего;
	\item[(b)] особого рода предложения, фиксирующие эмпирическое знание. \emph{(в философии науки)}
\end{itemize}

\subparagraph{Фатализм}
	философская концепция о существовании предопределенности высшей волей, роком, судьбой событий в природе, обществе и в жизни каждого человека

\subparagraph{Философские категории, во взаимоотношении представляют собой}
\begin{itemize}
	\item \emph{Содержание} --- подвижная динамическая сторона целого
	\item \emph{Форма} ---  система устойчивых связей предмета
\end{itemize}

\subparagraph{Целесообразность}
	свойство процессов и явлений приводить к определенному результату, цели в широком или условном смысле слова

\subparagraph{Синонимы системы и элемента}
\begin{itemize}
	\item[(a)] \emph{Целое} --- сущее, сохраняющее свое качество
	\item[(b)] \emph{Часть} ---  элемент внутри целого (вещь, свойство, отношение), обусловливающий сохранение качества
\end{itemize}

\subparagraph{Эволюция}
\begin{itemize}
	\item[a:] ~ синоним развития (в широком смысле)
	\item[b:] процессы изменения, протекающие в живой и неживой природе, а также в социальных системах
\end{itemize}

\subparagraph{Эклектизм}
	 метод, предполагающий заимствование принципов в нескольких различных системах и создание на их основе единой системы

\subparagraph{Эксперимент}
	род опыта, имеющего познавательный, целенаправленно исследовательский, методический характер, который проводится в специально заданных, воспроизводимых условиях путем их контролируемого изменения

\subparagraph{Элемент}
	понятие объекта, входящего в состав определенной системы и рассматриваемого в ее пределах как неделимый.

\subparagraph{Эмпиризм}
	 направление в теории познания, признающее чувственный опыт источником знаний и утверждающее, что все знание основывается на опыте

\subparagraph{Эсхатология}
	религиозное учение о конце истории и конечной судьбе мира

\subparagraph{Язык}
знаковая система, посредством которой осуществляется человеческое общение на самых различных уровнях, включая мышление, хранение и передачу информации и т. п








\end{document}
