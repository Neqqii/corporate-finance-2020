\documentclass[a4paper, 12pt]{article}

%%% Матпакет
\usepackage{amsmath, amssymb, amscd, amsthm, amsfonts}
\usepackage{hyperref}
\usepackage{icomma}                  % "Умная" запятая: $0,2$ --- число, $0, 2$ --- перечисление

%%% Страница
\usepackage{extsizes} % Возможность сделать 14-й шрифт
\usepackage{geometry} % Простой способ задавать поля
	\geometry{top=25mm}
	\geometry{bottom=25mm}
	\geometry{left=18mm}
	\geometry{right=14mm}
\usepackage{indentfirst}

%%%Стили
\usepackage{xcolor}
%\usepackage{sectsty}
%\allsectionsfont{\sffamily}
\usepackage{titlesec, blindtext, color} % подключаем нужные пакеты

%%% Работа с русским языком
\usepackage{cmap}					% поиск в PDF
\usepackage{mathtext} 				% русские буквы в фомулах
\usepackage[T2A]{fontenc}			% кодировка
\usepackage[utf8]{inputenc}			% кодировка исходного текста
\usepackage[english, russian]{babel}	% локализация и переносы

\usepackage{soul} % Модификаторы начертания
\usepackage{csquotes} % Цитаты

%%% Теоремы
\theoremstyle{plain} % Это стиль по умолчанию, его можно не переопределять.
\newtheorem{theorem}{Теорема}[section]
\newtheorem{proposition}[theorem]{Утверждение}
\newtheorem{definition}{Определение}

\theoremstyle{definition} % "Утверждение"
\newtheorem{corollary}{Следствие}
\newtheorem{problem}{Задача}[section]

\theoremstyle{remark} % "Примечание"
\newtheorem{example}{Пример}
\newtheorem{nota}{Примечание}

%%% Работа с картинками
\usepackage{graphicx}                % Для вставки рисунков
\graphicspath{{img/}}  				% папки с картинками
\setlength\fboxsep{3pt}              % Отступ рамки \fbox{} от рисунка
\setlength\fboxrule{1pt}             % Толщина линий рамки \fbox{}
\usepackage{wrapfig}                 % Обтекание рисунков текстом
\title{Философия \\ Вопросы к экзамену}
\author{ЭФ МГУ}
\date{зимняя сессия 2020}

%%% Работа с таблицами
\usepackage{array,tabularx,tabulary,booktabs} % Дополнительная работа с таблицами
\usepackage{longtable}                        % Длинные таблицы
\usepackage{multirow}                         % Слияние строк в таблице

\usepackage{hyperref}
\usepackage[usenames, dvipsnames, svgnames, table, rgb]{color}
\hypersetup{				                % Гиперссылки
    unicode=true,                           % русские буквы в раздела PDF
    pdftitle={Заголовок},                   % Заголовок
    pdfauthor={Автор},                      % Автор
    pdfsubject={Тема},                      % Тема
    pdfcreator={Создатель},                 % Создатель
    pdfproducer={Производитель},            % Производитель
    pdfkeywords={keyword1} {key2} {key3},   % Ключевые слова
    colorlinks=true,        	            % false: ссылки в рамках; true: цветные ссылки
    linkcolor=red,                          % внутренние ссылки
    citecolor=green,                        % на библиографию
    filecolor=magenta,                      % на файлы
    urlcolor=cyan                           % на URL
}

\begin{document}


\maketitle

\section{Специфика философского знания}
По свидетельству античных авторов (Диоген Лаэртский), слово "философия" встречается впервые у Пифагора, а в качестве названия особой сферы знания, термин "философия" впервые употреблялся Платоном. Зарождение философии исторически совпадает с возникновением зачатков научного знания, с появлением общественной потребности в целостном воззрении на мир и человека, в изучении общих принципов бытия и познания. Первые философы античного мира (Древняя Греция) стремились главным образом открыть единый источник многообразных природных явлений. Поэтому природа (по греч. фюсис; физика) стала первым всеобщим объектом, к которому обратилась философии, и в связи с этим сформировалось первое философское учение - натурфилософия (философия природы), которая явилась первой исторической формой философского мышления, первой философской дисциплиной в структуре философии (первыми физиологами были Фалес, Анаксимандр, Анаксимен, Гераклит).

\vspace{1em}
По мере накопления научно-философских знаний (природных, личностных и общественных) и выработки специальных приёмов исследования (аналитических - логических, математических) начался процесс специализации нерасчленённого знания, выделения математики, астрономии, медицины. Первоначально эти науки входили в состав философского знания. Но постепенно начинается и ограничение круга проблем, которыми занимается философия, происходит развитие, углубление, обогащение собственно философских представлений, возникают различные философские теории и течения. В ходе предметного самоопределения философии и её внутренней дифференциации (специализации) сформировались следующие философские дисциплины:
\subsection{Онтология}

\begin{definition}[Онтология]

\end{definition}
\end{document}
