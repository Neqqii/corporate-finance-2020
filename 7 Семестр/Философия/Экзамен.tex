\documentclass[a4paper, 12pt]{article}

%%% Матпакет
\usepackage{amsmath, amssymb, amscd, amsthm, amsfonts}
\usepackage{hyperref}
\usepackage{icomma}                  % "Умная" запятая: $0,2$ --- число, $0, 2$ --- перечисление

%%% Страница
\usepackage{extsizes} % Возможность сделать 14-й шрифт
\usepackage{geometry} % Простой способ задавать поля
	\geometry{top=25mm}
	\geometry{bottom=25mm}
	\geometry{left=18mm}
	\geometry{right=14mm}
\usepackage{indentfirst}

%%%Стили
\usepackage{xcolor}
%\usepackage{sectsty}
%\allsectionsfont{\sffamily}
\usepackage{titlesec, blindtext, color} % подключаем нужные пакеты

%%% Работа с русским языком
\usepackage{cmap}					% поиск в PDF
\usepackage{mathtext} 				% русские буквы в фомулах
\usepackage[T2A]{fontenc}			% кодировка
\usepackage[utf8]{inputenc}			% кодировка исходного текста
\usepackage[english, russian]{babel}	% локализация и переносы

\usepackage{soul} % Модификаторы начертания
\usepackage{csquotes} % Цитаты

%%% Теоремы
\theoremstyle{plain} % Это стиль по умолчанию, его можно не переопределять.
\newtheorem{theorem}{Теорема}[section]
\newtheorem{proposition}[theorem]{Утверждение}
\newtheorem{definition}{Определение}

\theoremstyle{definition} % "Утверждение"
\newtheorem{corollary}{Следствие}
\newtheorem{problem}{Задача}[section]

\theoremstyle{remark} % "Примечание"
\newtheorem{example}{Пример}
\newtheorem{nota}{Примечание}

%%% Работа с картинками
\usepackage{graphicx}                % Для вставки рисунков
\graphicspath{{img/}}  				% папки с картинками
\setlength\fboxsep{3pt}              % Отступ рамки \fbox{} от рисунка
\setlength\fboxrule{1pt}             % Толщина линий рамки \fbox{}
\usepackage{wrapfig}                 % Обтекание рисунков текстом
\title{Философия \\ Вопросы к экзамену}
\author{ЭФ МГУ}
\date{зимняя сессия 2020}

%%% Работа с таблицами
\usepackage{array,tabularx,tabulary,booktabs} % Дополнительная работа с таблицами
\usepackage{longtable}                        % Длинные таблицы
\usepackage{multirow}                         % Слияние строк в таблице

\usepackage{hyperref}
\usepackage[usenames, dvipsnames, svgnames, table, rgb]{color}
\hypersetup{				                % Гиперссылки
    unicode=true,                           % русские буквы в раздела PDF
    pdftitle={Заголовок},                   % Заголовок
    pdfauthor={Автор},                      % Автор
    pdfsubject={Тема},                      % Тема
    pdfcreator={Создатель},                 % Создатель
    pdfproducer={Производитель},            % Производитель
    pdfkeywords={keyword1} {key2} {key3},   % Ключевые слова
    colorlinks=true,        	            % false: ссылки в рамках; true: цветные ссылки
    linkcolor=red,                          % внутренние ссылки
    citecolor=green,                        % на библиографию
    filecolor=magenta,                      % на файлы
    urlcolor=cyan                           % на URL
}

\begin{document}


\maketitle

\section{Специфика философского знания}

По свидетельству античных авторов (Диоген Лаэртский), слово философия встречается впервые у Пифагора, а в качестве названия особой сферы знания, термин \emph{философия} впервые употреблялся Платоном. Зарождение философии исторически совпадает с возникновением зачатков научного знания, с появлением общественной потребности в целостном воззрении на мир и человека, в изучении общих принципов бытия и познания. Первые философы античного мира (Древняя Греция) стремились главным образом открыть единый источник многообразных природных явлений. Поэтому природа (по греч. фюсис; физика) стала первым всеобщим объектом, к которому обратилась философии, и в связи с этим сформировалось первое философское учение - \emph{натурфилософия} (философия природы), которая явилась первой исторической формой философского мышления, первой философской дисциплиной в структуре философии (первыми физиологами были Фалес, Анаксимандр, Анаксимен, Гераклит).

\vspace{1em}
По мере накопления научно-философских знаний (природных, личностных и общественных) и выработки специальных приёмов исследования (аналитических - логических, математических) начался процесс специализации нерасчленённого знания, выделения математики, астрономии, медицины. Первоначально эти науки входили в состав философского знания. Но постепенно начинается и ограничение круга проблем, которыми занимается философия, происходит развитие, углубление, обогащение собственно философских представлений, возникают различные философские теории и течения. В ходе предметного самоопределения философии и её внутренней дифференциации (специализации) сформировались следующие философские дисциплины:
\subsection{Основные дисциплины}

\subparagraph{Онтология}
    Учение о бытии, о принципах его строения, законах и формах.

    Проблема бытия понимается здесь в универсальном (всеохватывающем) смысле (\emph{почему есть нечто, а не ничто?} --- один из первых философских вопросов), анализируются бытие (сам принцип существования), небытиё (возможно ли несуществование, ничто), бытие материальное (природа) и идеальное (идея, мысль, дух). Часто онтология в истории философии отождествлялась с метафизикой (букв. то, что идет после физики), поскольку бытие рассматривалось не как то, что существует, а как то, что позволят всему существовать, само этим существующим не являясь;

\subparagraph{Гносеология и Эпистемология}
    Учение о познании; раздел философии, рассматривающий проблемы человеческого познания, вопросы о его возможностях и границах, о путях и средствах достижения истинного знания, о роли познания в бытии человека

\subparagraph{Логика}
    Наука о формах правильного (т.е. связного, последовательного, доказательного) мышления;

     Формами такого правильного (т.е. в соответствии с правилами логики, отражающей реальные процессы действительности) были понятия, суждения, умозаключения;
\subparagraph{Этика}
    Наука о нравственности, морали как формам общественного сознания, о должном и правильном поведении человека среди себе подобных;

    Проблемы этики связаны с проблемами свободы воли, природы нравственного поведения и его принципов, справедливости, добродетели, высшего блага, пользы и счастья как последствий человеческого поведения

\subparagraph{Эстетика}
    Наука о художественном и ценностном освоении мира: изучает принципы и условия творческого преобразования и
    изображения мира в таких категориях, как красота, прекрасное, возвышенное, безобразное, эстетическое восприятие, вкус, трагическое и комическое

\subparagraph{Философия истории} Освоение закономерностей исторического процесса

\subsection{subsection name}
Структуру философии в целом можно связать с такими тремя осваиваемыми ею явлениями, как \emph{истина, добро и красота} (и в связи с этим выделять в ней философские дисциплины).

\vspace{1em}
Начиная с эпохи Возрождения (14-16 века) процесс размежевания между философией и частными науками протекает всё более ускоренными темпами.
Возникает представление о естественной положительной науке, ограничивающейся преимущественно фактическими, опытными (эмпирическими) исследованиями. Специфику же философии теперь видят прежде всего в ее отвлечённости (спекулятивности, оторванности от опыта), абстрактности, умозрительности, теоретичности, которая очень часто вступает в противоречие с фактами и реалиями действительности. В это время философия еще продолжает заниматься общетеоретическими вопросами естественных наук. Но т.к философское исследование теоретических проблем частных наук не опиралось на достаточный для этой цели эмпирический материал, то оно носило умозрительный характер, и его результаты часто вступали в противоречие с фактами. На этой почве возникло противопоставление философии частным наукам. В 17-19 вв. создавались системы, в которых естествознанию противопоставлялась философия природы (противопоставлявшаяся физике), философия истории - истории как науке, философия права - правоведению. Считалось, что философия способна выходить за пределы опыта, давать "сверхопытное" знание. Но такого рода иллюзии опровергались дальнейшим развитием частных наук.

\vspace{1em}
В современном мире наука представляет собой разветвлённую систему знания. Все известные явления мира оказались в "частном" владении той или иной специальной науки (физики, биологии, географии). Однако в этой ситуации философия отнюдь не утрачивает своего значения (вопреки позитивизму, полагавшему, что с развитием естественных наук потребность в философии как отдельной от наук деятельности отпадает). Напротив, отказ от претензии на всезнание позволил философии более четко самоопределиться. Каждая наука исследует специфически определённую систему закономерностей, но ни одна частная наука не изучает закономерности, общие для явлении природы, развития общества и человеческого познания. Эти закономерности и являются предметом философии, определяя ее специфику.

\vspace{1em}
Уже в 20 веке внутри философии сформировались различные новые специальные дисциплины, связанные с характерным для философии обобщенным и абстрактным способом рассмотрения явлений действительности: философия языка (проблемы соотношения языка и действительности, языка и мысли, выразимости мыслей в языке), философия науки (проблемы научного знания, развития науки, методов науки), философская антропология (исследование наиболее общих и существенных вопросов человека, его природы и эволюции)

\end{document}
